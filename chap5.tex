\chapter{全纯函数的 Laurent 展开及其应用\label{chap5}}
前面已经证明,圆盘中的全纯函数一定可在该圆盘中展开成幂级数.现在问,圆环中的全纯函数是否也可以展开成幂级数?答案当然是否定的,因为幂级数是它的收敛圆中的全纯函数.本章中将证明,圆环中的全纯函数一定可以展开成Laurent级数,并由此得到一系列的新结果.

\section{全纯函数的Laurent展开}
称级数
\begin{equation}\label{eq5.1.1}
  \sum_{n=-\infty}^\infty a_n(z-z_0)^n = \sum_{n=0}^\infty a_n(z-z_0)^n + \sum_{n=1}^\infty
  a_{-n}(z-z_0)^{-n}
\end{equation}
为\textbf{Laurent级数}\index{L!Laurent级数}.它由两部分组成,第一部分就是幂级数,第二部分是负幂项的级数.如果这两个级数都收敛,就称级数 \eqref{eq5.1.1} 收敛.

我们首先关心的是级数 \eqref{eq5.1.1} 的收敛域的形状.

设第一个级数的收敛半径为$R$.对第二个级数作变换$\zeta=\frac1{z-z_0}$,它对$\zeta$而言就是幂级数
\[
  \sum_{n=1}^\infty a_{-n}(z-z_0)^{-n} = \sum_{n=1}^\infty a_{-n}\zeta^n.
\]
设其收敛半径为$\rho$,则当$|\zeta|<\rho$,或者$|z-z_0|>\frac1\rho$时,上述级数收敛.记$r=\frac1\rho$,则当$r<|z-z_0|<\infty$时,级数 \eqref{eq5.1.1} 中的负幂项级数收敛.
现在看在什么条件下级数 \eqref{eq5.1.1} 中的两个级数都能收敛.

如果$R\le r$,则当$|z-z_0|<R$时,必有$|z-z_0|<r$,这时级数 \eqref{eq5.1.1} 的第一个级数是收敛的,但第二个级数却发散了.当$|z-z_0|>r$时,必有$|z-z_0|>R$,这时级数 \eqref{eq5.1.1} 的第二个级数收敛而第一个级数发散.所以,两者不能同时收敛.

如果$r<R$,则当$r<|z-z_0|<R$时,级数 \eqref{eq5.1.1} 的两个级数都收敛,而且在这个圆环中内闭一致收敛,即级数 \eqref{eq5.1.1} 在上述圆环中内闭一致收敛,根据Weierstrass定理,它的和是圆环中的全纯函数. 这样,我们证明了下面的
\begin{theorem}\label{thm5.1.1}
  如果Laurent级数
  \begin{equation*}
    \sum_{n=-\infty}^\infty a_n(z-z_0)^n = \sum_{n=0}^\infty a_n(z-z_0)^n + \sum_{n=1}^\infty
  a_{-n}(z-z_0)^{-n}
  \end{equation*}
  的收敛域为圆环$D=\{z:r<|z-z_0|<R\}$,那么它在$D$中绝对收敛且内闭一致收敛,它的和函数在$D$中全纯..

  上述级数的幂级数部分称为该级数的\textbf{全纯部分}\index{L!Laurent级数!全纯部分},负幂项级数部分称为该级数的\textbf{主要部分}\index{L!Laurent级数!主要部分}.下面我们将看到,Laurent级数的一些重要性质取决于它的主要部分.
\end{theorem}

定理 \ref{thm5.1.1} 的逆定理也成立.
\begin{theorem}\label{thm5.1.2}
  设$D=\{z:r<|z-z_0|<R\}$,如果$f\in H(D)$,那么$f$在$D$上可以展开为Laurent级数:
  \begin{equation}\label{eq5.1.2}
    f(z) = \sum_{n=-\infty}^\infty a_n(z-z_0)^n,z\in D,
  \end{equation}
  其中,$a_n=\frac1{2\pi\ii}\int\limits_{\gamma_\rho}\frac{f(\zeta)}{(\zeta-z_0)^{n+1}}
  \dif \zeta$,而$\gamma_\rho=\{\zeta:|\zeta-z_0|=\rho\}$($r<\rho<R$),并且展开式 \eqref{eq5.1.2} 是唯一的.
\end{theorem}
\begin{proof}
  如图 \ref{fig5.1} 所示,任意取定$z\in D$,取$r_1,r_2$,使得
  \[
    r < r_1 < |z - z_0| < r_2 < R.
  \]
  记$\gamma_j=\{\zeta:|\zeta-z_0|=r_j\},j=1,2,$. 由第 \ref{chap3} 章 \ref{sec3.4} 节的定理 \ref{thm3.4.6},得
  \begin{equation}\label{eq5.1.3}
    f(z) = \frac1{2\pi\ii}\int\limits_{\gamma_2}\frac{f(\zeta)}{\zeta-z}\dif \zeta-
    \frac1{2\pi\ii}\int\limits_{\gamma_1}\frac{f(\zeta)}{\zeta-z}\dif \zeta.
  \end{equation}
  \begin{figure}[!ht]
    \centering
    \begin{tikzpicture}[thick,every node/.style={inner sep=2pt},
      >={Stealth[width=3pt]}]
      \draw(0,0)node[below right,inner sep=0pt]{$z_0$}circle(0.5)circle(0.8)
        circle(3)circle(3.5);
      \draw[very thin,->](3,0)--++(90:0.01)node[right]{$\gamma_2$};
      \draw[very thin,->](0.8,0)--++(90:0.01)node[right]{$\gamma_1$};
      \fill(0,0)circle(1pt);
      \draw(0,0)--node[below]{$R$}(35:3.5);
      \draw(0,0)--node[above]{$r$}(-0.5,0);
      \draw(0,0)--node[below right,fill=white,inner sep=0pt]{$r_1$}(-130:0.8);
      \draw(0,0)--node[right]{$r_2$}(85:3)(-40:1.6)node[right]{$z$};
      \fill(-40:1.6)circle(1pt);
    \end{tikzpicture}
    \caption{\label{fig5.1}}
  \end{figure}
  记$M_j=\sup\{|f(\zeta)|:\zeta\in\gamma_j\},j=1,2$.当$\zeta\in\gamma_1$时,
  $\bigg|\frac{\zeta-z_0}{z-z_0}\bigg|=\frac{r_1}{|z-z_0|}<1$,所以有
  \begin{align*}
    \frac1{\zeta-z} & = -\frac1{z-z_0} \bigg(1-\frac{\zeta-z_0}{z-z_0} \bigg)^{-1}\\
    & = -\sum_{n=0}^\infty\frac{(\zeta-z_0)^n}{(z-z_0)^{n+1}}=-
     \sum_{n=1}^\infty\frac{(\zeta-z_0)^{n-1}}{(z-z_0)^{n}},
  \end{align*}
  于是
  \begin{equation}\label{eq5.1.4}
    \frac{f(\zeta)}{\zeta-z} = -\sum_{n=1}^\infty f(\zeta)\frac{(\zeta-z_0)^{n-1}}{(z-z_0)^{n}}, \zeta\in\gamma_1.
  \end{equation}
  由于
  \[
    \bigg|\frac{f(\zeta)(\zeta-z_0)^{n-1}}{(z-z_0)^n}\bigg|
    \le\frac{M_1}{|z-z_0|}\bigg(\frac{r_1}{|z-z_0|}\bigg)^{n-1},
  \]
  并且右端是一收敛级数,故知级数 \eqref{eq5.1.4} 在$\gamma_1$上一致收敛,因而可逐项积分:
  \begin{equation}\label{eq5.1.5}
    \frac1{2\pi\ii}\int\limits_{\gamma_1}\frac{f(\zeta)}{\zeta-z}\dif \zeta
    = -\sum_{n=1}^\infty\bigg(\frac1{2\pi\ii}\int\limits_{\gamma_1}
    \frac{f(\zeta)}{(\zeta-z_0)^{-n+1}}\dif \zeta\bigg)(z-z_0)^{-n}.
  \end{equation}
  当$\zeta\in\gamma_2$时,$\bigg|\frac{z-z_0}{\zeta-z_0}\bigg|=\frac{|z-z_0|}{r_2}<1$,所以有
  \[
    \frac1{\zeta-z}=\frac1{\zeta-z_0}\bigg(1-\frac{z-z_0}{\zeta-z_0}\bigg)^{-1}
    =\sum_{n=0}^\infty\frac{(z-z_0)^n}{(\zeta-z_0)^{n+1}},
  \]
  于是
  \begin{equation}\label{eq5.1.6}
    \frac{f(\zeta)}{\zeta-z} = \sum_{n=0}^\infty f(\zeta)\frac{(z-z_0)^{n}}{(\zeta-z_0)^{n+1}},\zeta\in\gamma_2.
  \end{equation}
  与上面的讨论一样,级数 \eqref{eq5.1.6} 在$\gamma_2$上一致收敛,所以
  \begin{equation}\label{eq5.1.7}
    \frac1{2\pi\ii}\int\limits_{\gamma_2}\frac{f(\zeta)}{\zeta-z}\dif \zeta
    =\sum_{n=0}^\infty\bigg(\frac1{2\pi\ii}\int\limits_{\gamma_2}
    \frac{f(\zeta)}{(\zeta-z_0)^{n+1}}\dif \zeta\bigg)(z-z_0)^{n}.
  \end{equation}
  由多连通域的Cauchy积分定理,得
  \begin{align*}
    & \frac1{2\pi\ii}\int\limits_{\gamma_1}
      \frac{f(\zeta)}{(\zeta-z_0)^{-n+1}}\dif \zeta
    = \frac1{2\pi\ii}\int\limits_{\gamma_\rho}
      \frac{f(\zeta)}{(\zeta-z_0)^{-n+1}}\dif \zeta=a_{-n},\\
    & \frac1{2\pi\ii}\int\limits_{\gamma_2}
      \frac{f(\zeta)}{(\zeta-z_0)^{n+1}}\dif \zeta
    = \frac1{2\pi\ii}\int\limits_{\gamma_\rho}
      \frac{f(\zeta)}{(\zeta-z_0)^{n+1}}\dif \zeta=a_{n}.
  \end{align*}
  把它们分别代入 \eqref{eq5.1.5} 式和 \eqref{eq5.1.7} 式,得
  \begin{align*}
    & \frac1{2\pi\ii}\int\limits_{\gamma_1}\frac{f(\zeta)}{\zeta-z}\dif \zeta
    = -\sum_{n=1}^\infty a_{-n}(z-z_0)^{-n},\\
    & \frac1{2\pi\ii}\int\limits_{\gamma_2}\frac{f(\zeta)}{\zeta-z}\dif \zeta
    = \sum_{n=0}^\infty a_n(z-z_0)^n,
  \end{align*}
  再把它们代入 \eqref{eq5.1.3} 式,即得展开式 \eqref{eq5.1.2}.

  现在证明展开式 \eqref{eq5.1.2} 是唯一的.如果另有展开式
  \[
    f(z) = \sum_{n=-\infty}^{\infty}a_n'(z-z_0)^n,
  \]
  因为级数在$\gamma_\rho$上一致收敛,逐项积分得
  \[
    \frac1{2\pi\ii}\int\limits_{\gamma_\rho}
    \frac{f(\zeta)}{(\zeta-z_0)^{m+1}}\dif \zeta
    = \sum_{n=-\infty}^\infty a_n'\frac1{2\pi\ii}
    \int\limits_{\gamma_\rho}(\zeta-z_0)^{n-m-1}\dif \zeta=a_{m}'.
  \]
  所以这个展开式就是 \eqref{eq5.1.2} 式.
\end{proof}

\begin{example}\label{exam5.1.3}
  设$f(z)=\frac1{(z-1)(z-2)}$,试分别给出这个函数在$D_1=\{z:1<|z|<2\}$和$D_2=\{z:2<|z|<\infty\}$上的Laurent展开式.
\end{example}
\begin{solution}
  当$z\in D_1$时,由于$1<|z|<2$,所以
  \begin{align*}
    \frac1{(z-1)(z-2)} & = \frac1{z-2}-\frac1{z-1}\\
    & = -\frac12\frac1{1-\frac z2}-\frac1z\frac1{1-\frac1z}\\
    & = -\sum_{n=0}^\infty\frac1{2^{n+1}}z^n-\sum_{n=1}^\infty\frac1{z^n}.
  \end{align*}
  当$z\in D_2$时,由于$2<|z|<\infty$,所以
  \begin{align*}
    \frac1{(z-1)(z-2)} & = \frac1{z-2}-\frac1{z-1}
    = \frac1z \frac1{1-\frac2z} - \frac1z \frac1{1-\frac1z}\\
    & = \sum_{n=0}^\infty\frac{2^n}{z^{n+1}} - \sum_{n=0}^\infty\frac1{z^{n+1}}
    =\sum_{n=0}^\infty\frac{2^n-1}{z^{n+1}}. \qedhere
  \end{align*}
\end{solution}

\begin{xiti}
  \item 下列初等函数能否在指定的域$D$上展开为Laurent级数?
    \begin{enuma}
      \item $\cos\frac1z,D=B(0,\infty)\backslash\{0\}$;
      \item $\tan z,D=B(\infty,R),R>0$;
      \item $\Log\frac{z-1}{z-2},D=B(\infty,2)$;
      \item $\frac{z^2}{\sin\frac1z},D=B(0,r)\backslash\{0\},r>0$;
      \item $\Log\frac1{z-1},D=B(1,\infty)\backslash\{1\}$;
      \item $\sqrt[\leftroot{-1}\uproot{2}3]{(z-1)(z-2)(z-3)},D=B(\infty,3)$;
      \item $\sqrt{\frac z{(z-1)(z-2)}},D=B(0,2)\backslash\bar{B(0,1)}$;
      \item $\Log\frac{(z-1)(z-2)}{(z-3)(z-5)},D=B(0,3)\backslash\bar{B(0,2)}$.
    \end{enuma}
  \item 将下列初等函数在指定的域$D$上展开为Laurent级数:
    \begin{enuma}
      \item $\frac1{z^2(z-1)},D=B(1,1)\backslash\{1\}$;
      \item $\frac1{(z-1)(z-2)},D=B(0,2)\backslash\bar{B(0,1)}$;
      \item $\Log\bigg(\frac{z-1}{z-2}\bigg),D=B(\infty,2)$;
      \item $\sqrt{(z-1)(z-2)},D=B(\infty,2)$;
      \item $\frac1{(z-5)^n},n\in\MN,D=B(\infty,5)$;
      \item $\sin\frac z{1-z},D=B(1,\infty)\backslash\{1\}$;
      \item $\sqrt{\frac z{(z-1)(z-2)}},D=B(0,2)\backslash\bar{B(0,1)}$;
      \item $\ee^{\frac1{1-z}},D=B(\infty,1)$.
    \end{enuma}
  \item 将$\ee^{\frac z2\big(\zeta-\frac1\zeta\big)}$作为$\zeta$的函数在$B(0,\infty)\backslash\{0\}$上展开为Laurent级数
      \[
        \sum_{n=-\infty}^\infty J_n(z)\zeta^n,
      \]
      称$J_n(z)$($n\ge0$)为\textbf{Bessel函数}\index{B!Bessel函数}. 证明:
      \begin{align*}
        J_n(z) & = \frac1\pi\int_0^\pi\cos(n\theta-z\sin\theta)\dif \theta\\
        & = \sum_{k=0}^\infty\frac{(-1)^k}{k!(n+k)!}\bigg(\frac z2\bigg)^{n+2k},n\ge0.
      \end{align*}
  \item \hypertarget{xiti5.1.4}{} 设$0<r<R<\infty,D=B(0,R)\backslash\bar{B(0,r)}$. 证明:若$f(z)=\sum_{n=-\infty}^\infty a_nz^n$双全纯地将$D$映为域$G$,则$G$的面积为
      \[
        \pi\sum_{n=-\infty}^\infty n|a_n|^2(R^{2n}-r^{2n}).
      \]
  \item 证明(面积原理):若$f(z)=\frac1z+\sum_{n=1}^\infty a_nz^n$是$B(0,1)\backslash\{0\}$上的双全纯映射,则
      \[
        \sum_{n=1}^\infty n |a_n|^2 \le1.
      \]
  \item \hypertarget{xiti5.1.6}{} 证明:若$f(z)=z+\sum_{n=2}^\infty a_nz^n$是$B(0,1)$上的双全纯映射,则$|a_2|\le2$,并且$|a_2|=2$当且仅当$f(z)=\frac z{(1-\ee^{\ii\theta}z)^2},\theta\in\MR$.\\
     (\textbf{提示}:考虑$z\sqrt{\frac{f(z^2)}{z^2}}$的一个双全纯分支.)
  \item 证明:若$f(z)=z+\sum_{n=2}^\infty a_nz^n$是$B(0,1)$上的双全纯映射,则
      \begin{enuma}
        \item \label{xiti5.1.7.1}$\frac{1-|z|}{(1+|z|)^3}\le|f'(z)|\le\frac{1+|z|}{(1-|z|)^3},\forall z\in B(0,1)$;
        \item \label{xiti5.1.7.2}$\frac{|z|}{(1+|z|)^2}\le|f(z)|\le\frac{|z|}{(1-|z|)^2},\forall z\in B(0,1)$,特别地,$f\big(B(0,1)\big)\supset \frac14B(0,1)$.
        \item 上述不等式在异于零处成立,当且仅当$f(z)=\frac z{(1-\ee^{\ii\theta}z)^2},\theta\in\MR$.
      \end{enuma}
      (\textbf{说明}:不等式 \ref{xiti5.1.7.1} 称为偏差定理;不等式 \ref{xiti5.1.7.2} 称为增长定理和$\frac14$掩盖定理.)\\
      (\textbf{提示}:利用第 \hyperlink{xiti5.1.6}{6} 题的结论.)
  \item 证明:若$w=\varphi(z)$双全纯地将$D=\{z\in\MC:r_1<|z|<r_2\}$映为$G=\{w\in\MC:
      R_1<|w|<R_2\}$,其中,$0<r_1<r_2<\infty,0<R_1<R_2<\infty$,则$\frac{r_1}{r_2}=\frac{R_1}{R_2}$,并且$\varphi(z)=\ee^{\ii\theta}\frac{R_1z}{r_1}$或$\varphi(z)=\ee^{\ii\theta}\frac{R_2r_1}z,\theta\in\MR.$\\
      (\textbf{提示}:利用第 \hyperlink{xiti5.1.4}{4} 题的结论)
\end{xiti}

\section{孤立奇点\label{sec5.2}}
Laurent级数是研究全纯函数在孤立奇点附近性质的有力工具.

如果$f$在无心圆盘(即除去圆心后的圆盘)$\{z:0<|z-z_0|<R\}$中全纯,就称$z_0$是$f$的\textbf{孤立奇点}\index{G!孤立奇点}.

$f$在孤立奇点$z_0$附近可能有三种情形:
\begin{eenum}
  \item $\lim_{z\to z_0}f(z)=a,a$是一有限数,这时称$z_0$是$f$的\textbf{可去奇点}\index{G!孤立奇点!可去奇点};
  \item $\lim_{z\to z_0}f(z)=\infty$,这时称$z_0$是$f$的\textbf{极点}\index{G!孤立奇点!极点};
  \item $\lim_{z\to z_0}f(z)$不存在,这时称$z_0$是$f$的\textbf{本性奇点}\index{G!孤立奇点!本性奇点}.
\end{eenum}

现在我们分别讨论在上述三种情形下$f$在$z_0$点附近的性质. 先证明
\begin{theorem}[(\textbf{Riemann})]\label{thm5.2.1}\index{D!定理!Riemann定理}
  $z_0$是$f$的可去奇点的充分必要条件是$f$在$z_0$附近有界.
\end{theorem}
\begin{proof}
  必要性是显然的,因为如果$z_0$是$f$的可去奇点,那么$\lim_{z\to z_0}f(z)=a$,$f$在$z_0$附近当然有界.现在设$f$在$z_0$附近有界,即存在$\varepsilon>0$,使得当$z$满足$0<|z-z_0|<\varepsilon$时,$|f(z)|<M$. 因为$f$在无心圆盘$D=\{z:0<|z-z_0|<R\}$中全纯,根据定理 \ref{thm5.1.2},$f$在$D$中有Laurent展开式:
  \begin{equation}\label{eq5.2.1}
    f(z) = \sum_{n=-\infty}^\infty a_n(z-z_0)^n,z\in D,
  \end{equation}
  其中,$a_n=\frac1{2\pi\ii}\int\limits_{\gamma_\rho}\frac{f(\zeta)}{(\zeta-z_0)^{n+1}}
  \dif \zeta,0<\rho<R,\gamma_{\rho}=\{\zeta:|\zeta-z_0|=\rho\}$. 今取$0<\rho<\varepsilon$,故当$\zeta\in\gamma_\rho$时,$|f(\zeta)|<M$. 于是,由长大不等式得
  \[
    |a_{-n}| = \bigg|\frac1{2\pi\ii}\int\limits_{\gamma_\rho}
    \frac{f(\zeta)}{(\zeta-z_0)^{-n+1}}\dif \zeta\bigg|\le
    \frac M{2\pi\rho^{-n+1}}\cdot2\pi\rho=M\rho^n,
  \]
  让$\rho\to0$,即得$a_{-n}=0,n=1,2,\cdots$. 这说明在$f$的Laurent展开式 \eqref{eq5.2.1} 中,所有负次幂的系数都是零,因而展开式 \eqref{eq5.2.1} 是一个幂级数. 所以$\lim_{z\to z_0}f(z)=a_0$,即$z_0$是一个可去奇点.
\end{proof}

从上面的证明可以看出,在$z_0$是$f$的可去奇点的情形下,$f$在$\{z:0<|z-z_0|<R\}$中的展开式为
\[
  f(z) = \sum_{n=0}^\infty a_n(z-z_0)^n,
\]
只要令$f(z_0)=a_0$,上式便在圆盘$B(z_0,R)$中成立了,因而$f$在$z_0$处全纯.换句话说,在这种情形下,只要适当定义$f$在$z_0$处的值,便能使$f$在$z_0$处全纯.这就是称$z_0$为$f$的可去奇点的原因.

再看极点的情形,先证明
\begin{prop}\label{prop5.2.2}
  $z_0$是$f$的极点的充分必要条件是$z_0$为$\frac1f$的零点.
\end{prop}
\begin{proof}
  如果$z_0$是$f$的极点,即$\lim_{z\to z_0}f(z)=\infty$,那么存在$\varepsilon>0$,使得当$0<|z-z_0|<\varepsilon$时,$f(z)$不等于零. 故$\varphi(z)=\frac1{f(z)}$在上述无心圆盘中全纯,且$\lim_{z\to z_0}\varphi(z)=0$,即$z_0$是$\varphi$的可去奇点,且$\varphi(z_0)=0$. 反之,如果$z_0$是$\varphi(z)=\frac1{f(z)}$的零点,则
  \[
    \lim_{z\to z_0}f(z) = \lim_{z\to z_0}\frac1{\varphi(z)} = \infty,
  \]
  即$z_0$是$f$的极点.
\end{proof}

\begin{definition}\label{def5.2.3}
  如果$z_0$是$\frac1{f(z)}$的$m$阶零点,就称$z_0$是$f$的$m$阶极点.
\end{definition}

现在可以证明
\begin{theorem}\label{thm5.2.4}
  $z_0$是$f$的$m$阶极点的充分必要条件是$f$在$z_0$附近的Laurent展开式为
  \begin{equation}\label{eq5.2.2}
    f(z) = \frac{a_{-m}}{(z-z_0)^m} + \cdots+\frac{a_{-1}}{z-z_0} + a_0 + a_1(z-z_0) + \cdots,
  \end{equation}
  其中,$a_{-m}\ne0$.
\end{theorem}
\begin{proof}
  如果$z_0$是$f$的$m$阶极点,根据定义,它是$\frac1f$的$m$阶零点.由第 \ref{chap4} 章的命题 \ref{prop4.3.4},它在$z_0$的邻域中可以表示为$\frac1{f(z)}=(z-z_0)^mg(z)$,这里,$g$在$z_0$处全纯,且$g(z_0)\ne0$,因而$\frac1g$也在$z_0$处全纯.设$\frac1g$在$z_0$处的Taylor展开为
  \[
    \frac1{g(z)} = \sum_{n=0}^\infty c_n(z-z_0)^n,
  \]
  这里,$c_0\ne0$,于是
  \begin{align*}
    f(z) & = \frac1{(z-z_0)^m}\frac1{g(z)} = \sum_{n=0}^\infty c_n(z-z_0)^{n-m}\\
    & = \frac{c_0}{(z-z_0)^m} + \cdots + \frac{c_{m-1}}{z-z_0}
      + c_m + c_{m+1}(z-z_0) + \cdots.
  \end{align*}
  记$a_n=c_{n+m},n=-m,\cdots,-1,0,1,\cdots$,即得展开式 \eqref{eq5.2.2}.

  反之,如果$f$在$z_0$附近的Laurent展开式为 \eqref{eq5.2.2} 式,那么
  \[
    (z-z_0)^mf(z) = a_{-m}+a_{-(m-1)}(z-z_0) + \cdots + a_0(z-z_0)^m + \cdots.
  \]
  若记上式右端的幂级数为$\varphi(z)$,则$\varphi$在$z_0$处全纯,且$\varphi(z_0)=a_{-m}\ne0$. 因而$\frac1{\varphi}$也在$z_0$处全纯,于是
  \[
    \frac1{f(z)} = (z-z_0)^m\frac1{\varphi(z)}
  \]
  在$z_0$附近成立. 由第 \ref{chap4} 章的命题 \ref{prop4.3.4},$z_0$是$\frac1f$的$m$阶零点,所以是$f$的$m$阶极点.
\end{proof}

最后来看$f$在本性奇点附近的性质.前面已经证明,$f$在可去奇点处的特征是Laurent展开式没有主要部分,只有全纯部分;在极点处的特征是Laurent展开式的主要部分只有有限项.因此,在本性奇点处的特征是Laurent展开式的主要部分有无穷多项.按定义,$z_0$是$f$的本性奇点是指$\lim_{z\to z_0}f(z)$不存在,实际上,我们可以证明一个更深刻的结果.
\begin{theorem}[(\textbf{Weierstrass})]\label{thm5.2.5}\index{D!定理!Weierstrass定理}
  设$z_0$是$f$的本性奇点,那么对任意$A\in\MC_\infty$,必存在趋于$z_0$的点列$\{z_n\}$,使得$\lim_{n\to\infty}f(z_n)=A$.
\end{theorem}
\begin{proof}
  先设$A=\infty$.因为$z_0$是$f$的本性奇点,故$f$在$z_0$附近无界.于是对任意自然数$n$,总能找到$z_n$,使得$|z_n-z_0|<\frac1n$,但$|f(z_n)|>n$,这说明$\lim_{n\to\infty}f(z_n)=\infty$.

  再设$A$是一个有限数.令$\varphi(z)=\frac1{f(z)-A}$,我们证明$\varphi$在$z_0$的邻域中无界.不然的话,$z_0$是$\varphi$的可去奇点,适当选择$\varphi(z_0)$的值,可使$\varphi$在$z_0$处全纯. 如果$\varphi(z_0)\ne0$,则因$f(z)=\frac1{\varphi(z)}+A$,$f$也在$z_0$处全纯,这不可能.故必有$\varphi(z_0)=0$,这说明$z_0$是$f$的极点,也不可能.所以,$\varphi$在$z_0$的邻域中无界.于是,对任意自然数$n$,存在$z_n$,使得$|z_n-z_0|<\frac1n$,但$\frac1{|f(z)-A|}>n$,即$|f(z)-A|<\frac1n$.这就证明了$\lim_{n\to\infty}f(z_n)=A$.
\end{proof}

后来,Picard又证明了比Weierstrass定理更进一步的结果:
\begin{theorem}[(\textbf{Picard})]\label{thm5.2.6}\index{D!定理!Picard定理}
  全纯函数在本性奇点的邻域内无穷多次地取到每个有穷复值,最多只有一个例外.
\end{theorem}

例如,考虑函数$f(z)=\ee^{\frac1z}$,它在$z=0$附近是全纯的.若让$z$沿着$x$轴分别从$0$的左边和右边趋于$0$,可得
\begin{align*}
  & \lim_{z=x\to0^-}\ee^{\frac1z} = \lim_{x\to0^-}\ee^{\frac1x} = 0,\\
  & \lim_{z=x\to0^+}\ee^{\frac1z} = \lim_{x\to0^+}\ee^{\frac1x} = \infty.
\end{align*}
这说明$\lim_{z\to0}\ee^{\frac1z}$不存在,所以$z=0$是$\ee^{\frac1z}$的本性奇点.对于任意复数$a\ne0$,若取$z_n=(\log a+2n\pi\ii)^{-1}$,则$f(z_n)=\ee^{\log a+2n\pi\ii}=a$.由于$z_n\to0$,这说明$\ee^{\frac1z}$在$z=0$的邻域中可以无穷多次地取到非零值$a$,但$0$是它的唯一的例外值.

这个定理的证明超出本书的范围,因此从略.

上面讨论的是孤立奇点为有限点的情形,现在讨论无穷远点为孤立奇点的情形.

如果$f$在无穷远点的邻域(不包括无穷远点)$\{z:0\le R<|z|<\infty\}$中全纯,就称$\infty$是$f$的孤立奇点.

在这种情形下,作变换$z=\frac1\zeta$,记
\[
  g(\zeta) = f\bigg(\frac1\zeta\bigg),
\]
则$g$在$0<|\zeta|<\frac1R$中全纯,即$\zeta=0$是$g$的孤立奇点. 很自然地,我们有下面的
\begin{definition}\label{def5.2.7}
  如果$\zeta=0$是$g$的可去奇点、$m$阶极点或本性奇点,那么我们相应地称$z=\infty$是$f$的可去奇点、$m$阶极点或本性奇点.
\end{definition}

因为$g$在原点的邻域中有Laurent展开:
\[
  g(\zeta) = \sum_{n=-\infty}^\infty a_n\zeta^n ,0 < |\zeta| < \frac1R,
\]
所以$f$在$R<|z|<\infty$中有下面的Laurent展开:
\[
  f(z) = \sum_{n=-\infty}^\infty b_nz^n,
\]
其中,$b_n=a_{-n},n=0,\pm1,\cdots$.

特别地,如果$z=\infty$是$f$的可去奇点,即$\zeta=0$是$g$的可去奇点,因而$a_n=0$($n=-1,-2,\cdots$),所以$f$的Laurent展开为
\begin{equation}\label{eq5.2.3}
  f(z) = \sum_{n=0}^\infty b_{-n}z^{-n}.
\end{equation}

同样道理,如果$z=\infty$分别是$f$的$m$阶极点或本性奇点,那么$f$在$R<|z|<\infty$中分别有下面的Laurent展开式:
\begin{equation}\label{eq5.2.4}
  f(z) = b_mz^m + \cdots + b_1 + b_0 + b_{-1}z^{-1} + \cdots,
\end{equation}
或
\begin{equation}\label{eq5.2.5}
  f(z) = \cdots + b_mz^m + \cdots + b_1z + b_0 + b_{-1}z^{-1} + \cdots.
\end{equation}
这时,我们称$\sum_{n=1}^\infty b_nz^n$为$f$的主要部分,$\sum_{n=0}^\infty b_{-n}z^{-n}$为$f$的全纯部分.

\begin{xiti}
  \item 是否存在$\bar{B(0,1)}\backslash\{0\}$上的无界全纯函数$f$,使得$\lim_{z\to0}zf(z)=0$?
  \item 下列初等全纯函数有哪些奇点?指出其类别:
    \begin{tasks}(2)
      \task $\frac1{\sin z-\cos z}$;
      \task $\frac{\ee^{\frac1{1-z}}}{\ee^z-1}$;
      \task $\sin\frac1{1-z}$;
      \task $\tan z$;
      \task $\frac{\ee^z}{z(1-\ee^z)}$;
      \task $\ee^{\cot\frac1z}$;
      \task $\sin\bigg(\frac1{\cos\frac1z}\bigg)$;
      \task $\ee^{\tan z}$.
    \end{tasks}
  \item 证明:若$z_0$是全纯函数$f:B(z_0,r)\backslash\{z_0\}\to\MC\backslash\{0\}$的本性奇点,则$z_0$也是$\frac1{f(z)}$的本性奇点.
  \item 设$R(z)$是有理函数,$z_1,z_2,\cdots,z_n$是$R(z)$在$\MC_\infty$上的全部不同的极点.证明:若$z_0$是全纯函数$f:B(z_0,r)\backslash\{z_0\}\to\MC_\infty\backslash
    \{z_1,\cdots,z_n\}$的本性奇点,则$z_0$也是$R\big(f(z)\big)$的本性奇点.
  \item 设$P_n(z)$和$Q_m(z)$分别是$n$次和$m$次多项式,指出$\infty$是下列有理函数的什么奇点:
    \begin{enuma}
      \item $P_n(z)+Q_m(z)$;
      \item $\frac{P_n(z)}{Q_m(z)}$;
      \item $P_n(z)Q_m(z)$.
    \end{enuma}
  \item 设$f$是$B(z_0,R)\backslash\{z_0\}$上非常数的全纯函数.证明:若$z_0$是$f$的零点集的极限点,则$z_0$是$f$的本性奇点.
  \item 若$f$在域$D$上除了极点外,在其他点处都全纯,则称$f$是$D$上的\textbf{亚纯函数}\index{F!复变函数!亚纯函数}.证明:若$f$是$B(z_0,R)\backslash\{z_0\}$上的亚纯函数,并且$z_0$是$f$的极点集的极限点,则对任意$A\in \MC_\infty$,存在收敛于$z_0$的点列$\{z_n\}\subset B(z_0,R)\backslash\{z_0\}$,使得$\lim_{n\to\infty}f(z_n)=A$.
  \item 设$f$在$B(0,R)\backslash\{0\}$上全纯. 证明:若$\Re f(z)>0,\forall z\in B(0,R)\backslash\{0\}$,则$0$是$f$的可去奇点.
  \item 设$f$是域$D$上的亚纯函数.证明:对于任意$A\in \MC$,
      $\frac{f'(z)}{f(z)-A}$也是$D$上的亚纯函数,并且其极点都是$1$阶的($f$为常值函数的情形除外).
  \item 证明:若$f$是域$D$上的亚纯函数,但不全纯,则存在$R>0$,使得$B(\infty,R)\subset f(D)$.
  \item 设$f$在$\MC\backslash\{0\}$上全纯,并且$0$和$\infty$都是$f$的本性奇点.证明:若令$A(r)=\max_{|z|=r}\Re f(z)$, $0<r<\infty$,则
      \begin{align*}
        & \lim_{r\to\infty}\frac{\log A(r)}{\log r} = \infty,\\
        & \lim_{r\to0}\frac{\log A(r)}{\log\frac1r} = \infty.
      \end{align*}
\end{xiti}

\section{整函数与亚纯函数\label{sec5.3}}
前面已经讲过,如果$f$在整个复平面$\MC$上全纯,就称$f$为整函数,$f$在$\MC$上有Taylor展开式
\begin{equation}\label{eq5.3.1}
  f(z) = \sum_{n=0}^\infty a_nz^n.
\end{equation}
它当然在$R<|z|<\infty$中也成立,因此也可把它看成是无穷远点邻域中的Laurent展开式.

如果整函数$f$在$\infty$处全纯,那么根据 \ref{sec5.2} 节中的展开式 \eqref{eq5.2.3},它在$\infty$处邻域中的Laurent展开式除去常数项外只有负次幂的项,因此在展开式 \eqref{eq5.3.1} 中必须有
\[
  a_1 = a_2 = \cdots = 0,
\]
所以$f$是一常数.这样,我们就得到了
\begin{theorem}\label{thm5.3.1}
  在无穷远处全纯的整函数一定是常数.
\end{theorem}

如果无穷远点是整函数$f$的一个$m$阶极点,那么根据 \ref{sec5.2} 节中的展开式 \eqref{eq5.2.4},它在无穷远点邻域中的Laurent展开式除去一个$m$次多项式外只有负次幂的项,因此在展开式 \eqref{eq5.3.1} 中必须有
\[
  a_{m+1} = a_{m+2} = \cdots = 0.
\]
所以$f$是一个$m$次多项式.我们已经证明了下面的
\begin{theorem}\label{thm5.3.2}
  如果无穷远点是整函数$f$的一个$m$阶极点,那么$f$是一个$m$次多项式.
\end{theorem}

不是常数和多项式的整函数称为超越整函数.无穷远点一定是超越整函数的本性奇点.如$\ee^z,\sin z,\cos z$等,都是超越整函数.

如果$f$在整个复平面$\MC$上除去极点外没有其他的奇点,就称$f$是一个\textbf{亚纯函数}\index{F!复变函数!亚纯函数}.整函数当然是亚纯函数.此外,有理函数
\[
  f(z) = \frac{P_n(z)}{Q_m(z)}
\]
也是亚纯函数,这里,$P_n$和$Q_m$是两个既约的多项式.现在来看有理函数在无穷远点的情况.若记
\begin{align*}
  & P_n(z) = a_0 + a_1z + \cdots + a_nz^n, a_n \ne 0,\\
  & Q_m(z) = b_0 + b_1z + \cdots + b_mz^m, b_m \ne 0,
\end{align*}
那么
\[
  f(z) = \frac{P_n(z)}{Q_m(z)} = \frac1{z^{m-n}}\frac{a_n+a_{n-1}\frac1z+\cdots+a_0\frac1{z^n}}
  {b_m+b_{m-1}\frac1z+\cdots+b_0\frac1{z^m}}.
\]
所以
\[
  \lim_{z\to\infty}f(z) = \begin{cases}
     \frac{a_n}{b_m}, & n = m; \\
       \infty, & n > m; \\
        0, & n < m.
   \end{cases}
\]
这说明$z=\infty$或是$f$的可去奇点,或是$f$的极点.下面我们将证明,这一事实的逆也成立.
\begin{theorem}\label{thm5.3.3}
  若$z=\infty$是亚纯函数$f$的可去奇点或极点,则$f$一定是有理函数.
\end{theorem}
\begin{proof}
  因$z=\infty$是$f$的可去奇点或极点,故必存在$R>0$,使得$f$在$R<|z|<\infty$中全纯.在$|z|\le R $中,$f$最多只能有有限个极点.因若有无穷多个极点$z_j,j=1,2,\cdots$,则$\{z_j\}$必有收敛的子列$\{z_{k_j}\}$,设其极限为$a$,则$|a|\le R$,显然$a$不是孤立奇点,这不可能.今设$z_1,\cdots,z_n$为$f$在$|z|\le R$中的有限个极点,它们的阶分别为$m_1,\cdots,m_n$. $f$在$z_j$($j=1,\cdots,n$)附近的Laurent展开的主要部分为
  \[
    h_j(z) = \frac{c_{-1}^{(j)}}{z-z_j} + \frac{c_{-2}^{(j)}}{(z-z_j)^2}
    + \cdots + \frac{c_{-m_j}^{(j)}}{(z-z_j)^{m_j}}.
  \]
  设$f$在$\infty$的邻域内的Laurent展开的主要部分为$g$,当$z=\infty$是$f$的极点时,$g$是一个多项式;当$z=\infty$是$f$的可去奇点时,$g\equiv0$.令
  \[
    F(z) = f(z) - h_1(z) - \cdots - h_n(z) - g(z),
  \]
  显然,$F$在$\MC_\infty$中除$z_1,\cdots,z_n$和$\infty$外是全纯的,而在$z_1,\cdots,z_n$和$\infty$这些点上,$f$的主要部分都已经被消去,因而也是全纯的.所以,$F$是$\MC_\infty$上的全纯函数,因而由定理 \ref{thm5.3.1},$F$是一个常数$c$.于是
  \[
    f(z) = c + g(z) + \sum_{j=1}^n h_j(z),
  \]
  所以$f$是有理函数.
\end{proof}

这里,我们顺便得到了这样一个结论:任何有理函数一定能分解成部分分式之和,而且这种分解是唯一的.这个结论在计算有理函数的不定积分时已经被多次用过.

作为上面三个定理的应用,我们可以定出$\MC$的全纯自同构群和$\MC_\infty$的亚纯自同构群.
\begin{theorem}\label{thm5.3.4}
  $\Aut(\MC)$由所有的一次多项式组成.
\end{theorem}
\begin{proof}
  设$f(z)=az+b,a\ne0$,则显然$f\in \Aut(\MC)$.反之,对于任意的$f\in \Aut(\MC)$,因为$f$是整函数,如果$\infty$是它的可去奇点,则由定理 \ref{thm5.3.1},$f$是一个常数,这不可能.如果$\infty$是$f$的本性奇点,则由定理 \ref{thm5.2.5},对于任意$A\in\MC$,必有$z_n\to\infty$,使得$\lim_{n\to\infty}f(z_n)=A$.现在记$f(z_n)=w_n$.,则$z_n=f^{-1}(w_n)$,两端令$n\to\infty$,即得$f^{-1}(A)=\infty$.这说明$A$是$f^{-1}$的一个极点,与$f^{-1}$是整函数相矛盾.由此可知$\infty$必为$f$的极点,由定理 \ref{thm5.3.2} 知道,$f$是一个多项式.又因为$f$在$\MC$上是单叶的,所以$f$只能是一次多项式.
\end{proof}

\begin{theorem}\label{thm5.3.5}
  $\Aut(\MC_\infty)$由所有的分式线性变换组成.
\end{theorem}
\begin{proof}
  因为是在$\MC_\infty$上讨论,$\Aut(\MC)$中的元素不再是全纯函数,而是亚纯函数.由第 \ref{chap2} 章 \ref{sec2.5} 节的讨论知道,任何分式线性变换都是$\Aut(\MC_\infty)$中的元素.现设$f\in\Aut(\MC_\infty)$,则$f$必为亚纯函数,而且$\infty$必是$f$的可去奇点或极点.由定理 \ref{thm5.3.3},$f$必为有理函数,再由它的单叶性,它只能是分式线性变换.
\end{proof}

\begin{xiti}
  \item 求出所有$\MC$上的亚纯函数$f$,使得$|f(z)|=1,\forall z\in \partial B(0,1)$.
  \item 证明:整函数$f(z)$无零点,当且仅当存在另一个整函数$g(z)$,使得$f(z)=\ee^{g(z)}$.
  \item 设$P_n(z)$是$n$次多项式,$n\in\MN$. 证明:$\ee^z-P_n(z)$有无数个零点.
  \item 设$\SL(2,\MC)=\left\{\begin{pmatrix}
                           a & b \\
                           c & d
                           \end{pmatrix}:ad - cb = 1\right\},
                           I = \begin{pmatrix}
                           1 & 0 \\
                           0 & 1
                           \end{pmatrix}$.证明:
    \begin{enuma}
      \item $\SL(2,\MC)$按矩阵乘法构成一个群,$\{ I,- I\}$是其正规子群;
      \item $\begin{pmatrix}
        a & b \\
        c & d
       \end{pmatrix}\mapsto\frac{az+b}{cz+d}$是商群$\frac{\SL(2,\MC)}{\{ I,- I\}}$与$\MC_\infty$的自同构群$\Aut(\MC_\infty)$之间的同构.
    \end{enuma}
  \item 设$f(z)$是整函数. 证明:
    \begin{enuma}
      \item 若$f(\MR)\subset \MR,f(\ii\MR)\subset \ii\MR$,则$f(z)$是奇函数;
      \item 若$f(\MR)\subset \MR,f(\ii\MR)\subset \MR$,则$f(z)$是偶函数.
    \end{enuma}
  \item 设$f$在$\MC_\infty$上亚纯,其极点只有$z=1,z=2$和$z=\infty$. 若$f$在这$3$个极点处的Laurent展开式的主要部分分别为$\frac1{z-1},\frac1{z-2}+\frac1{(z-2)^2}$和$z+z^2$,并且$f(0)=0$,求$f(z)$.
\end{xiti}

\section{残数定理\label{sec5.4}}
设$f$在$a$点全纯,那么对于$a$点邻域中的任意可求长闭曲线$\gamma$,都有$\int\limits_\gamma f(z)\dz=0$. 如果$a$是$f$的孤立奇点,那么上述积分不一定总等于零,且积分值只与$f$和$a$有关,而与$\gamma$无关. 现在来计算这个积分. 设$f$在$a$点邻域中的Laurent展开式为
\[
  f(z) = \sum_{n=-\infty}^\infty c_n(z-a)^n,
\]
这里,
\[
  c_n = \frac1{2\pi\ii}\int\limits_{\gamma}\frac{f(\zeta)}{(\zeta-a)^{n+1}}\dif \zeta,n=0,\pm1,\cdots.
\]
特别地,当$n=-1$时,我们有
\begin{equation}\label{eq5.4.1}
  c_{-1} = \frac1{2\pi\ii}\int\limits_{\gamma}f(\zeta)\dif \zeta.
\end{equation}
原来所讨论的积分值就是$c_{-1}$的$2\pi\ii$倍,因此$c_{-1}$这个系数有它特殊的含义.我们给出下面的
\begin{definition}\label{def5.4.1}
  设$a$是$f$的一个孤立奇点,$f$在$a$点的邻域$B(a,r)$中的Laurent展开为$f(z)=\sum_{n=-\infty}^\infty c_n(z-a)^n$,称$c_{-1}$为$f$在$a$点的\textbf{残数}\index{C!残数}
  \footnote{残数亦称留数.},记为
  \[
    \Res(f,a) = c_{-1}
  \]
  或
  \[
    \Res_{z=a}f = c_{-1}.
  \]
\end{definition}

根据 \eqref{eq5.4.1} 式,我们有
\begin{equation}\label{eq5.4.2}
  \int\limits_\gamma f(z)\dz = 2\pi\ii\Res(f,a).
\end{equation}
这里,$\gamma=\{z:|z-a|=\rho\},0<\rho<r$.

若$z=\infty$是$f$的孤立奇点,即$f$在$R<|z|<\infty$中全纯,我们定义$f$在$z=\infty$处的残数为
\begin{equation}\label{eq5.4.3}
  \Res(f,\infty) = -\frac1{2\pi\ii}\int\limits_\gamma f(z)\dz,
\end{equation}
这里,$\gamma=\{z:|z|=\rho\},R<\rho<\infty$.

在很多情况下,函数在孤立奇点处的Laurent展开式是不易得到的,因此有必要讨论在不知道Laurent展开式的情况下计算残数的方法.

\begin{prop}\label{prop5.4.2}
  若$a$是$f$的$m$阶极点,则
  \[
    \Res(f,a) = \frac1{(m-1)!}\lim_{z\to a}\frac{\dif ^{m-1}}{\dif z^{m-1}}
    \{(z-a)^mf(z)\}.
  \]
\end{prop}
\begin{proof}
  因为$a$是$f$的$m$阶极点,故在$a$点的邻域中有
  \begin{equation}\label{eq5.4.4}
    f(z) = \frac1{(z-a)^m}g(z),
  \end{equation}
  这里,$g$在$a$点全纯,且$g(a)\ne0$. 于是
  \[
    f(z) = \frac1{(z-a)^m}\sum_{n=0}^\infty\frac{g^{(n)}(a)}{n!}(z-a)^n
    =\sum_{n=0}^\infty\frac{g^{(n)}(a)}{n!}(z-a)^{n-m}.
  \]
  这是一个Laurent展开式,$(z-a)^{-1}$的系数为$\frac{g^{(m-1)}(a)}{(m-1)!}$. 由 \eqref{eq5.4.4} 式知$g(z)=(z-a)^mf(z)$,因而得
  \begin{equation*}
    \Res(f,a) = \frac{g^{(m-1)}(a)}{(m-1)!} = \frac1{(m-1)!}\lim_{z\to a}\frac{\dif ^{m-1}}{\dif z^{m-1}}\{(z-a)^mf(z)\}. \qedhere
  \end{equation*}
\end{proof}

特别地,当$m=1$时,我们有下面的
\begin{prop}\label{prop5.4.3}
  若$a$是$f$的$1$阶极点,则
  \[
    \Res(f,a) = \lim_{z\to a}(z-a)f(z).
  \]
\end{prop}

\begin{example}\label{exam5.4.4}
  若$f(z)=\frac1{1+z^2},z=\pm\ii$都是$f$的$1$阶极点,由命题 \ref{prop5.4.3} 即得
  \begin{align*}
    & \Res(f,\ii) = \lim_{z\to \ii}(z-\ii)\frac1{1+z^2} = \frac1{2\ii},\\
    & \Res(f,-\ii) = \lim_{z\to-\ii}(z+\ii)\frac1{1+z^2} = -\frac1{2\ii}.
  \end{align*}
  在某些情况下,下面的命题用起来更方便.
\end{example}

\begin{prop}\label{prop5.4.5}
  设$f=\frac gh,g$和$h$都在$a$处全纯,且$g(a)\ne0,h(a)=0,h'(a)\ne0$,那么
  \[
    \Res(f,a) = \frac{g(a)}{h'(a)}.
  \]
\end{prop}
\begin{proof}
  在所设的条件下,$a$是$f$的$1$阶极点,故由命题 \ref{prop5.4.3} 即知道
  \begin{equation*}
    \Res(f,a) = \lim_{z\to a}(z-a)\frac{g(z)}{h(z)}
    = \lim_{z\to a}\frac{g(z)}{\frac{h(z)-h(a)}{z-a}} = \frac{g(a)}{h'(a)}. \qedhere
  \end{equation*}
\end{proof}

\begin{example}\label{exam5.4.6}
  计算$f(z)=\frac{\ee^z}{\sin z}$在$z=0$处的残数.
\end{example}
\begin{solution}
  这时$g(z)=\ee^z,h(z)=\sin z$. 于是$g(0)=1,h(0)=0,h'(0)=1$,因而由命题 \ref{prop5.4.5} 得
  \begin{equation*}
    \Res(f,0) = 1. \qedhere
  \end{equation*}
\end{solution}

\begin{example}
  计算函数$f(z)=\frac{\ee^{\ii z}}{z(z^2+1)^2}$在$z=-\ii$处的残数.
\end{example}
\begin{solution}
  显然,$z=-\ii$是$f$的一个$2$阶极点,利用命题 \ref{prop5.4.2},得
  \begin{equation*}
    \Res(f,-\ii) = \lim_{z\to-\ii}\dd{}z\bigg(\frac{\ee^{\ii z}}{z(z-\ii)^2}\bigg)
    = \frac{\ee}4. \qedhere
  \end{equation*}
\end{solution}

如果$a$是$f$的本性奇点,就没有像上面那种简单的计算残数的公式了,这时只能通过$f$的Laurent展开来得到$f$在$a$点的残数.

\begin{example}\label{exam5.4.8}
  计算$f(z)=\ee^{z+\frac1z}$在$z=0$处的残数.
\end{example}
\begin{solution}
  因为
  \[
    f(z) = \ee^z\cdot\ee^{\frac1z} =
   \bigg(1 + z + \frac{z^2}{2!} + \cdots \bigg)\bigg(1 + \frac1z + \frac1{2!z^2} + \cdots \bigg),
  \]
  这个乘积中,$\frac1z$的系数为
  \[
    1 + \frac1{2!} + \frac1{2!3!} + \frac1{3!4!} + \cdots,
  \]
  这就是要找的残数,即
  \begin{equation*}
    \Res(f,a) = \sum_{n=0}^\infty\frac1{n!(n+1)!}. \qedhere
  \end{equation*}
\end{solution}

残数理论的基本定理是下面的
\begin{theorem}\label{thm5.4.9}
  设$D$是复平面上的一个有界区域,它的边界$\gamma$由一条或若干条简单闭曲线组成.如果$f$在$D$中除去孤立奇点$z_1,\cdots,z_n$外是全纯的,在闭域$\bar D$上除去$z_1,\cdots,z_n$外是连续的,那么
  \begin{equation}\label{eq5.4.5}
    \int\limits_\gamma f(z)\dz = 2\pi\ii\sum_{k=1}^n\Res(f,z_k).
  \end{equation}
\end{theorem}
\begin{proof}
  在$D$内以$z_k$($k=1,2,\cdots,n$)为中心作一小圆周$\gamma_k$,使得所有$\gamma_k$都在$D$的内部,且每一个$\gamma_k$都在其余小圆周的外部. 于是由定理 \ref{thm3.2.5},得
  \[
    \int\limits_\gamma f(z)\dz = \sum_{k=1}^n\int\limits_{\gamma_k}f(z)\dz.
  \]
  再由公式 \eqref{eq5.4.2},即得所要证的公式 \eqref{eq5.4.5}.
\end{proof}

这个定理称为\textbf{残数定理}\index{D!定理!残数定理},它的主要贡献是把积分计算归结为残数的计算. 而从命题 \ref{prop5.4.2} 知道,计算残数是一个微分运算. 因此,从实质上来说,残数定理把积分运算变成了微分运算,从而带来了方便.

\begin{example}\label{exam5.4.10}
  计算积分
  \[
    \int\limits_{\gamma} \frac{z}{(z^2-1)^2(z^2+1)} \dz,
  \]
  这里,$\gamma=\{z:|z-1|=\sqrt3\}$.
\end{example}
\begin{proof}被积函数
  \[
    f(z) = \frac{z}{(z^2-1)^2(z^2+1)}
  \]
  有两个$1$阶极点$z_1=\ii,z_2=-\ii$,以及两个$2$阶极点$z_3=1,z_4=-1$. 容易看出,$z_1,z_2,z_3$都在$\gamma$的内部,$z_4$在$\gamma$的外部. 由残数定理得
  \[
    \int\limits_\gamma f(z)\dz = 2\pi\ii\sum_{k=1}^3\Res(f,z_k).
  \]
  由命题 \ref{prop5.4.3} 和命题 \ref{prop5.4.2},得
  \begin{align*}
    & \begin{aligned}
        \Res(f,\ii) & = \lim_{z\to\ii}(z-\ii)f(z) \\
        & = \lim_{z\to\ii}\frac{z}{(z^2-1)^2(z+\ii)}=\frac18,
      \end{aligned}\\
    & \begin{aligned}
        \Res(f,-\ii) & = \lim_{z\to-\ii}(z+\ii)f(z)\\
        & = \lim_{z\to-\ii}\frac{z}{(z^2-1)^2(z-\ii)} = \frac18,
      \end{aligned}\\
    & \begin{aligned}
        \Res(f,1) & = \lim_{z\to1} \dd{}z\{(z-1)^2f(z)\} \\
        & = \lim_{z\to1}\dd{}z\bigg\{\frac z{(z+1)^2(z^2+1)}\bigg\}\\
        & = \lim_{z\to1}\frac{-3z^3-z^2-z+1}{(z+1)^3(z^2+1)^2}\\
        & = -\frac18.
      \end{aligned}
  \end{align*}
    因而有
    \[
      \int\limits_\gamma f(z)\dz=2\pi\ii\bigg(\frac18 + \frac18 - \frac18\bigg)
      = \frac{\pi\ii}4. \qedhere
    \]
\end{proof}

\begin{example}\label{exam5.4.11}
  计算积分
  \[
    \int\limits_{|z|=1} \frac{z^2\sin^2z}{(1-\ee^z)^5}\dz.
  \]
\end{example}
\begin{solution}
  容易看出,被积函数
  \[
    f(z) = \frac{z^2\sin^2z}{(1-\ee^z)^5}
  \]
  在$|z|=1$内只有一个极点$z=0$. 对于这种类型的函数,直接从Laurent展开来求残数更方便些:
  \[
    \frac{z^2\sin^2z}{(1-\ee^z)^5}
    = \frac{z^2\big(z-\frac{z^3}{3!}+\cdots\big)^2}{\big(-z-\frac{z^2}{2!}-\cdots\big)^5}
    = -\frac{z^4\big(1-\frac{z^2}{3!}+\cdots\big)^2}{z^5\big(1+\frac z{2!}+\cdots\big)^5}.
  \]
  因为$\frac{\big(1-\frac{z^2}{3!}+\cdots\big)^2}{\big(1+\frac z{2!}+\cdots\big)^5}$在$z=0$处全纯,且在$z=0$处等于$1$,故其Taylor展开可写为$1+c_1z+\cdots$,于是得
  \[
    \frac{z^2\sin^2z}{(1-\ee^z)^5} = -\frac1z(1+c_1z+\cdots),
  \]
  因而$\Res(f,0)=-1$. 由残数定理即得
  \begin{equation*}
    \int\limits_{|z|=1}\frac{z^2\sin^2z}{(1-\ee^z)^5}\dz = -2\pi\ii. \qedhere
  \end{equation*}
\end{solution}

残数定理也可写成下面的形式:
\begin{theorem}\label{thm5.4.12}
  若$f$在$\MC$中除去$z_1,\cdots,z_n$外是全纯的,则$f$在$z_1,\cdots,z_n$及$z=\infty$处的残数之和为零.
\end{theorem}
\begin{proof}
  取$R$充分大,使得$z_1,\cdots,z_n$都在$B(0,R)$中.于是,由残数定理得
  \begin{equation}\label{eq5.4.6}
    \int\limits_{|z|=R}f(z)\dz=2\pi\ii\sum_{k=1}^n\Res(f,z_k).
  \end{equation}
  但由 \eqref{eq5.4.3} 式得
  \begin{equation}\label{eq5.4.7}
    -\int\limits_{|z|=R}f(z)\dz=2\pi\ii\Res(f,\infty).
  \end{equation}
  由 \eqref{eq5.4.6} 式和 \eqref{eq5.4.7} 式即得所要证之结论.
\end{proof}

\begin{xiti}
  \item 证明:残数定理与Cauchy积分公式等价.
  \item 若$a$是$B(a,R)\backslash\{a\}$上全纯函数$f$的可去奇点,其中$a\ne\infty$,则显然$\Res(f,a)=0$.举例说明,若$\infty$是$B(\infty,R)$上全纯函数$f$的可去奇点,则$\Res(f,\infty)$可能不等于零.
  \item 设$f\in H\big(B(\infty,R)\big)$. 证明:
    \begin{enuma}
      \item 若$\infty$是$f$的可去奇点,则
        \[
          \Res(f,\infty) = \lim_{z\to\infty}z^2f'(z);
        \]
      \item 若$\infty$是$f$的$m$阶极点,则
        \[
          \Res(f,\infty) = \frac{(-1)^m}{(m+1)!}\lim_{z\to\infty}z^{m+2}f^{(m+1)}(z).
        \]
    \end{enuma}
  \item 设$f,g\in H\big(B(a,r)\big),f(a)\ne0,a$是$g$的$2$阶零点,计算$\Res\bigg(\frac fg,a\bigg)$.
  \item 设$f$在$\MC$上除去孤立奇点外,在其他点处都全纯.证明:
    \begin{enuma}
      \item 若$f$是偶函数,则$\Res(f,-a)=-\Res(f,a)$;
      \item 若$f$是奇函数,则$\Res(f,-a)=\Res(f,a)$.
    \end{enuma}
  \item 设$D$是由有限条可求长简单闭曲线围成的域,$g\in H(D)\cap C(\bar D)$. 证明:若
    \begin{enuma}
      \item $f$在$D$上亚纯,在$D$中的全部彼此不同的极点为$w_1,w_2,\cdots,w_m$,其相应的阶数分别为
        $q_1,q_2,\cdots,q_m$;
      \item $f$在$\bar D\backslash\{w_1,w_2,\cdots,w_m\}$上全纯,在$\partial D$上没有零点;
      \item $f$在$D$中的全部彼此不同的零点为$z_1,z_2,\cdots,z_n$,其相应的阶数分别为$p_1,p_2,\cdots,p_n$,
    \end{enuma}
    则
    \[
      \frac1{2\pi\ii}\int\limits_{\partial D}g(z)\frac{f'(z)}{f(z)}\dz = \sum_{j=1}^n
      p_jg(z_j)-\sum_{j=1}^m q_jg(w_j).
    \]
    并说明这是Cauchy积分公式和辐角原理的推广.
  \item 求下列初等函数在指定点的残数:
    \begin{enuma}
      \item $\Res\bigg(\frac{\sin\alpha z}{z^3\sin\beta z},0\bigg)$($\alpha\ne\beta,\beta\ne0$);
      \item $\Res\bigg(\frac1{(1+z^2)^{n+1}},\ii\bigg)$($n\in\MN$);
      \item $\Res\bigg(\Log\frac{z-a}{z-b},\infty\bigg)$($a\ne b$);
      \item $\Res\bigg(\frac1{z^2}\ee^{\frac1z}\Log\frac{1-\alpha z}{1-\beta z},0\bigg)
           $($\alpha\ne\beta$);
      \item $\Res\bigg(z^3\cos\frac1{z-2},2\bigg)$;
      \item $\Res(\cot^3z,0)$;
      \item $\Res\bigg(\frac1{(z-a)^n(z-b)^m},a\bigg)$($a\ne b,m,n\in\MN$);
      \item $\Res\bigg(\frac{z^{2n}}{(1+z)^n},\infty\bigg)$($n\in\MN$).
    \end{enuma}
  \item 指出下列初等函数在$\MC_\infty$中的全部孤立奇点,并求出这些初等函数在它们各自孤立奇点处的残数:
    \begin{tasks}(2)
      \task $\frac1{z^3-z^5}$;
      \task $\frac{z^3+z^2+2}{z(z^2-1)^2}$;
      \task $\frac{z^2+z-1}{z^2(z-1)}$;
      \task $\frac{z^{n-1}}{z^n+a^n}$($a\ne0,n\in\MN$);
      \task $\frac1{\sin z}$;
      \task $\sin\frac z{z+1}$;
      \task $\frac{\ee^z}{z(z-1)}$;
      \task $\frac{\ee^{\pi z}}{z^2+1}$.
    \end{tasks}
  \item 设$f,g\in H\big(B(0,R)\big)\cap C\big(\bar{B(0,R)}\big),g$在$\partial B(0,R)$上无零点,$g$在$B(0,R)$中的全部零点$z_1,z_2,\cdots,z_n$都是$1$阶零点,求
      \[
        \frac1{2\pi\ii}\int\limits_{|z|=R}\frac{f(z)}{zg(z)}\dz.
      \]
  \item 求积分:
    \begin{enuma}
      \item $\int\limits_{|z|=2}\frac1{z^3(z^{10}-2)}\dz$;
      \item $\int\limits_{|z|=1}\frac1{(z-a)^n(z-b)^n}\dz$($|a|<1<|b|,n\in\MN$);
      \item $\int_0^{2\pi}\ee^{\cos\theta}[\cos(n\theta-\sin\theta)+\ii
          \sin(n\theta-\sin\theta)]\dif \theta$($n\in\MZ$);
      \item $\int\limits_{|z|=R}\frac{z^2\dz}{\ee^{2\pi\ii z^3}-1}$($n<R^3<n+1,n\in\MN$).
    \end{enuma}
  \item 计算积分
    \begin{enuma}
      \item $\int\limits_{|z|=R}\sqrt{(z-a)(z-b)}\dz$ $\bigg(a\ne b,R>\max\{|a|,|b|\},
        \lim_{z\to\infty}\frac{\sqrt{(z-a)(z-b)}}z=1\bigg)$;
      \item $\int\limits_{|z|=R}z^n\log\frac{z-a}{z-b}\dz$ $\big(a\ne b,R>\max\{|a|,|b|\}\big)$.
    \end{enuma}
  \item 设$D$是由有限条可求长简单闭曲线围成的域,$f(z)$在$D$上亚纯,在$D$中的全部彼此不同的极点为$w_1,w_2,\cdots,w_m$,其相应的Laurent展开式的主要部分为$f_1(z),f_2(z),\cdots$,$f_m(z)$,并且在$\bar D\backslash\{w_1,w_2,\cdots,w_m\}$上连续.证明:对于任意$z\in D$,有
      \[
        \frac1{2\pi\ii}\int\limits_{\partial D}\frac{f(\zeta)}{\zeta-z}\dif \zeta=f(z)-\sum_{j=1}^mf_j(z).
      \]
\end{xiti}

\section{利用残数定理计算定积分\label{sec5.5}}
残数定理的重要应用之一是计算各种定积分.在微积分的课程中我们已经知道,大部分函数的原函数不能用初等函数来表达,因此,通过求原函数来计算定积分只对一部分函数有效,很多定积分的计算要想其他的办法来解决,利用残数定理就是重要的方法之一.它的基本思想是这样的:为了求实函数$f(x)$在实轴上或实轴上某一区间$I$上的积分,我们在$I$上适当地加一辅助曲线$l$,使其与$I$构成一简单闭曲线$\gamma$,其内部记为$D$.同时适当选取复变函数$F(z)$(当然是根据$f(x)$来选取),然后在$\bar D$上对$F(z)$应用残数定理,这样就把要求的积分转化为计算$F(z)$在$D$内奇点处的残数和$l$上的积分了.当然,$l$和$F(z)$的选取是富于技巧的.

我们分几种类型来讨论:

\subsection{\mbox{$\int_{-\infty}^\infty f(x){\normalfont\dif }x$}型积分}
先证明下面的
\begin{theorem}\label{thm5.5.1}
  设$f$在上半平面$\{z:\Im z>0\}$中除去$a_1,\cdots,a_n$外是全纯的,在$\{z:\Im z\ge0\}$中除去$a_1,\cdots,a_n$外是连续的.如果$\lim_{z\to\infty}zf(z)=0$,那么
  \begin{equation}\label{eq5.5.1}
    \int_{-\infty}^\infty f(x)\dx = 2\pi\ii\sum_{k=1}^n\Res(f,a_k).
  \end{equation}
  \begin{figure}[!ht]
    \centering
    \begin{tikzpicture}[thick,every node/.style={inner sep=2pt},scale=1.4,
      >={Stealth[width=3pt]}]
      \draw(-2.5,0)--(-2,0)node[below]{$-R$}--(0,0)node[below]{$O$}
         --(2,0)node[below]{$R$}--(2.5,0);
      \draw(2,0)arc(0:180:2);
      \draw[->,very thin](2,0)arc(0:40:2)node[above right]{$\gamma_R$};
      \fill(-2,0)circle(0.7pt)(2,0)circle(0.7pt)(0,0)circle(0.7pt)
        (60:1.5)circle(0.7pt)node[right]{$a_1$}(85:1.7)circle(0.7pt)
        (100:1.54)circle(0.7pt)node[above]{$a_2$}(150:0.8)circle(0.7pt)node[above right]{$a_n$}
        (145:1.4)circle(0.7pt);
    \end{tikzpicture}
    \caption{\label{fig5.2}}
  \end{figure}
\end{theorem}
\begin{proof}
  如图 \ref{fig5.2} 所示,取充分大的$R$,使得$a_1,\cdots,a_n$包含在半圆盘$\{z:|z|<R,\Im z>0\}$中,记$\gamma_R=\{z:z=R\ee^{\ii\theta},0\le\theta\le\pi\}$,由残数定理得
  \begin{equation}\label{eq5.5.2}
    \int_{-R}^Rf(x)\dx+\int\limits_{\gamma_R}f(z)\dz=2\pi\ii\sum_{k=1}^n\Res(f,a_k).
  \end{equation}
  记$M(R)=\max\{|f(z)|:z\in \gamma_R\}$,由假定,$\lim_{R\to\infty}RM(R)=0$,因而
  \[
    \bigg|\int\limits_{\gamma_R}f(z)\dz\bigg|=\bigg|\int_0^\pi f(R\ee^{\ii\theta})R\ii\ee^{\ii\theta}\dif \theta\bigg|
    \le\pi RM(R)\to0\;\mbox{($R\to\infty$)}.
  \]
  在 \eqref{eq5.5.2} 式中令$R\to\infty$,即得公式 \eqref{eq5.5.1}.
\end{proof}

\begin{corollary}\label{cor5.5.2}
  设$P$和$Q$是两个既约多项式,$Q$没有实的零点,且$\deg Q-\deg P\ge2$,那么
  \[
    \int_{-\infty}^\infty\frac{P(x)}{Q(x)}\dx=2\pi\ii\sum_{k=1}^n
    \Res\bigg(\frac{P(z)}{Q(z)},a_k\bigg),
  \]
这里,$a_k$($k=1,\cdots,n$)为$Q$在上半平面中的全部零点,$\deg P,\deg Q$分别为$P$和$Q$的次数.
\end{corollary}
\begin{proof}
  令$f(z)=\frac{P(z)}{Q(z)}$,则$f$满足定理 \ref{thm5.5.1} 的条件,由定理 \ref{thm5.5.1} 即得本推论.
\end{proof}
\begin{example}\label{exam5.5.3}
  计算积分
  \[
    \int_{-\infty}^\infty \frac{x^2-x+2}{x^4+10x^2+9} \dx.
  \]
\end{example}
\begin{solution}
  令$f(z)=\frac{z^2-z+2}{z^4+10z^2+9}$,它满足推论 \ref{cor5.5.2} 的条件. 容易看出,分母$Q(z)=z^4+10z^2+9$有$4$个零点$\pm\ii$和$\pm3\ii$,但在上半平面中的零点只有$a_1=\ii$和$a_2=3\ii$两个,容易算得
  \begin{align*}
    & \Res(f,\ii) = \frac{-1-\ii}{16},\\
    & \Res(f,3\ii) = \frac{3-7\ii}{48},
  \end{align*}
  故得
  \begin{equation*}
    \int_{-\infty}^\infty\frac{x^2-x+2}{x^4+10x^2+9}\dx = \frac5{12}\pi. \qedhere
  \end{equation*}
\end{solution}

\begin{example}\label{exam5.5.4}
  计算积分
  \[
    \int_{-\infty}^\infty\frac{\dx}{(1+x^2)^{n+1}}.
  \]
\end{example}
\begin{solution}
  令$f(z)=\frac1{(1+z^2)^{n+1}}$,它显然满足推论 \ref{cor5.5.2} 的条件,且在上半平面中只有一个$n+1$阶极点$z=\ii$. 应用命题 \ref{prop5.4.2},通过直接计算得
  \[
    \Res(f,\ii) = \frac1{2\ii}\frac{(2n)!}{2^{2n}(n!)^2},
  \]
  于是得
  \begin{equation*}
    \int_{-\infty}^\infty\frac{\dx}{(1+x^2)^{n+1}} = \frac{(2n)!\pi}{2^{2n}(n!)^2}. \qedhere
  \end{equation*}
\end{solution}

在计算$\int_{-\infty}^\infty\ee^{\ii\alpha x}f(x)\dx$这种类型的积分时,需要应用下面的Jordan引理:
\begin{lemma}[(\textbf{Jordan})]\label{lemma5.5.5}\index{D!定理!Jordan引理}
  设$f$在$\{z:R_0\le|z|<\infty,\Im z\ge0\}$上连续,且$\lim_{\substack{z\to\infty\\
  \Im z\ge0}}f(z)$\\$=0$,则对任意$\alpha>0$,有
  \[
    \lim_{R\to\infty}\int\limits_{\gamma_R}\ee^{\ii\alpha z}f(z)\dz = 0,
  \]
  这里,$\gamma_R=\{z:z=R\ee^{\ii\theta},0\le\theta\le\pi,R\ge R_0\}$.
\end{lemma}
\begin{proof}
  记$M(R)=\max\{|f(z)|:z\in\gamma_R\}$,则由假定,$M(R)\to0$($R\to\infty$). 因为
  \[
    \int\limits_{\gamma_R}\ee^{\ii\alpha z}f(z)\dz = \int_0^\pi\ee^{\ii\alpha R\cos\theta}\ee^{-\alpha R\sin\theta}f(R\ee^{\ii\theta})R\ii\ee^{\ii\theta}\dif \theta,
  \]
  所以
  \begin{align*}
    \bigg| \int\limits_{\gamma_R}\ee^{\ii\alpha z}f(z) \dz \bigg|
    & \le RM(R)\int_0^\pi\ee^{-\alpha R\sin\theta}\dif \theta
    = 2RM(R)\int_0^{\frac\pi2}\ee^{-\alpha R\sin\theta}\dif \theta\\
    & \le 2RM(R)\int_0^{\frac\pi2}\ee^{-\frac2\pi \alpha R\theta} \dif \theta
    = \frac\pi\alpha M(R)(1-\ee^{-\alpha R})\to0\;\mbox{($R\to\infty$)}.
  \end{align*}
  这里,我们已经利用了不等式
  \begin{equation*}
    \sin\theta \ge \frac2\pi\theta \left(0\le\theta\le\frac\pi2\right). \qedhere
  \end{equation*}
\end{proof}

现在可以证明下面的
\begin{theorem}\label{thm5.5.6}
  设$f$在上半平面$\{z:\Im z>0\}$中除去$a_1,\cdots,a_n$外是全纯的,在$\{z:\Im z\ge0\}$中除去
  $a_1,\cdots,a_n$外是连续的. 如果$\lim_{z\to\infty}f(z)=0$,那么对任意$\alpha>0$,有
  \begin{equation}\label{eq5.5.3}
    \int_{-\infty}^\infty \ee^{\ii\alpha x}f(x)\dx=2\pi\ii\sum_{k=1}^n\Res\big(
    \ee^{\ii\alpha z}f(z),a_k\big).
  \end{equation}
\end{theorem}
\begin{proof}
  取充分大的$R$,使得$a_1,\cdots,a_n$都包含在半圆盘$\{z:|z|<R,\Im z>0\}$中,对函数
  \[
    F(z) = \ee^{\ii\alpha z}f(z)
  \]
  用残数定理,得
  \begin{equation}\label{eq5.5.4}
    \int_{-R}^R\ee^{\ii\alpha x}f(x)\dx+\int\limits_{\gamma_R}\ee^{\ii\alpha z}f(z)\dz
    = 2\pi\ii\sum_{k=1}^n\Res\big(\ee^{\ii\alpha z}f(z),a_k\big),
  \end{equation}
  根据Jordan引理,有
  \[
    \lim_{R\to\infty}\int\limits_{\gamma_R}\ee^{\ii\alpha z}f(z)\dz = 0.
  \]
  在 \eqref{eq5.5.4} 式的两端让$R\to\infty$,即得公式 \eqref{eq5.4.3}.
\end{proof}

注意到
\[
  \ee^{\ii\alpha x} = \cos\alpha x+\ii\sin\alpha x,
\]
在公式 \eqref{eq5.5.3} 的两端分别取实部和虚部,即得
\begin{corollary}\label{cor5.5.7}
  在定理 \ref{thm5.5.6} 的条件下,我们有
  \begin{align*}
    & \int_{-\infty}^\infty f(x)\cos \alpha x\dx = \Re\bigg\{2\pi\ii\sum_{k=1}^n\Res\big(
    \ee^{\ii\alpha z}f(z),a_k\big)\bigg\},\\
    & \int_{-\infty}^\infty f(x)\sin \alpha x\dx=\Im\bigg\{2\pi\ii\sum_{k=1}^n\Res\big(
    \ee^{\ii\alpha z}f(z),a_k\big)\bigg\}.\\
  \end{align*}
\end{corollary}

\begin{example}
  计算积分
  \[
    \int_{-\infty}^\infty\frac{\cos a x}{b^2+x^2}\dx\;\mbox{($a>0,b>0$)}.
  \]
\end{example}
\begin{solution}
  令$f(z)=\frac1{b^2+z^2}$,它满足定理 \ref{thm5.5.6} 的条件. 因为$\frac{\ee^{\ii a z}}{b^2+z^2}$在上半平面中只有一个$1$阶极点$z=b\ii$,且
  \[
   \Res\bigg(\frac{\ee^{\ii a z}}{b^2+z^2},b\ii\bigg) = \frac{\ee^{-ab}}{2b\ii},
  \]
  根据推论 \ref{cor5.5.7},即得
  \begin{equation*}
    \int_{-\infty}^\infty\frac{\cos a x}{b^2+x^2}\dx = \frac\pi b\ee^{-ab}. \qedhere
  \end{equation*}
\end{solution}

遇到$f$在实轴上有奇点的情况时,常要使用下面的引理:
\begin{lemma}\label{lemma5.5.9}
  设$f$在扇状域
  \[
    G = \{z=a+\rho\ee^{\ii\theta}:0<\rho\le\rho_0,\theta_0\le\theta\le\theta_0+\alpha\}
  \]
  上连续,如果$\lim_{z\to a}(z-a)f(z)=A$,那么
  \begin{equation}\label{eq5.5.5}
    \lim_{\rho\to0}\int\limits_{\gamma_\rho}f(z)\dz=\ii A\alpha,
  \end{equation}
  这里,$\gamma_\rho=\{z=a+\rho\ee^{\ii\theta}:\theta_0\le\theta\le\theta_0+\alpha\}$,它的方向是沿着辐角增加的方向.
\end{lemma}
\begin{proof}
  令$g(z)=(z-a)f(z)-A$,则$\lim_{z\to a}g(z)=0$. 若记$M_\rho=\sup\{|g(z)|:z=a+\rho\ee^{\ii\theta},\theta_0\le\theta\le\theta_0+\alpha\}$,则$\lim_{\rho\to0}M_\rho=0$. 于是
  \[
    \bigg|\int\limits_{\gamma_\rho}\frac{g(z)}{z-a}\dz\bigg|
    =\bigg|\int_{\theta_0}^{\theta_0+\alpha}\frac{g(a+\rho\ee^{\ii\theta})}
    {\rho\ee^{\ii\theta}}\rho\ii\ee^{\ii\theta}\dif \theta\bigg|
    \le M_\rho\alpha\to0\;\mbox{($\rho\to0$).}
  \]
  由此即得
  \begin{equation*}
    \int\limits_{\gamma_\rho}f(z)\dz=\ii A\alpha+\int\limits_{\gamma_\rho}
    \frac{g(z)}{z-a}\dz\to\ii A\alpha\;\mbox{($\rho\to0$)}. \qedhere
  \end{equation*}
\end{proof}

\begin{example}\label{exam5.5.10}
  计算积分
  \[
    \int_0^\infty\frac{\sin x}x\dx.
  \]
\end{example}
\begin{figure}[!ht]
  \centering
  \begin{tikzpicture}[thick,every node/.style={inner sep=2pt},scale=1.3,
    >={Stealth[width=3pt]}]
    \draw(-3,0)node[below]{$-R$}--(-1,0)node[below]{$-\rho$}arc(180:0:1)node[below]{$\rho$}
      --(3,0)node[below]{$R$}arc(0:180:3);
    \draw[->,very thin](0,1)node[above]{$\gamma_\rho$}--++(0.01,0);
    \draw[->,very thin](0,3)node[above]{$\gamma_R$}(45:3)--++(-0.01,0.01);
    \fill(0,0)circle(0.7pt)node[below]{$O$};
  \end{tikzpicture}
  \caption{\label{fig5.3}}
\end{figure}
\begin{solution}
  取函数$f(z)=\frac{\ee^{\ii z}}z$,取围道如图 \ref{fig5.3} 所示,它由线段$[-R,-\rho]$,$[\rho,R]$和半圆周$\gamma_\rho,\gamma_R$组成.在此围道围成的域中,$f$是全纯的,因而由Cauchy积分定理得
  \begin{equation}\label{eq5.5.6}
    \int_{-R}^{-\rho}\frac{\ee^{\ii x}}x\dx+\int\limits_{\gamma_\rho^-}\frac{\ee^{\ii z}}z\dz+\int_\rho^R\frac{\ee^{\ii x}}x\dx+\int\limits_{\gamma_R}\frac{\ee^{\ii z}}z\dz=0.
  \end{equation}
  由Jordan引理知道
  \[
    \lim_{R\to\infty}\int\limits_{\gamma_R}\frac{\ee^{\ii z}}z\dz = 0.
  \]
  由引理 \ref{lemma5.5.9} 得
  \[
    \lim_{\rho\to0}\int\limits_{\gamma_\rho^-}\frac{\ee^{\ii z}}z\dz = -\ii\pi.
  \]
  在 \eqref{eq5.5.6} 式中令$\rho\to0,R\to\infty$,于是得
  \[
    \int_{-\infty}^0\frac{\ee^{\ii x}}x\dx+\int_0^\infty\frac{\ee^{\ii x}}x\dx = \ii\pi,
  \]
  即
  \[
    \int_{-\infty}^\infty\frac{\ee^{\ii x}}x\dx = \ii\pi.
  \]
  两边取虚部,得
  \[
    \int_{-\infty}^\infty\frac{\sin x}x\dx = \pi,
  \]
  因而
  \begin{equation*}
    \int_0^\infty\frac{\sin x}x\dx = \frac12\int_{-\infty}^\infty\frac{\sin x}x\dx = \frac\pi2. \qedhere
  \end{equation*}
\end{solution}

注意,如果$f$是偶函数,$\int_0^\infty f(x)\dx$的值可以通过等式
\[
  \int_0^\infty f(x)\dx = \frac12\int_{-\infty}^\infty f(x)\dx
\]
从$\int_{-\infty}^\infty f(x)\dx$的值得到,就像上面的例子那样.如果$f$不是偶函数,这个方法就不行了.下面就来讨论计算$\int_0^\infty f(x)\dx$的方法.

\subsection{\mbox{$\int_0^\infty f(x){\normalfont\dif }x$}型积分\protect\hypertarget{sec5.5.2}{}}

用残数定理计算$\int_0^\infty f(x)\dx$这种类型的积分,往往要借助于对数函数,不像计算$\int_{-\infty}^\infty f(x)\dx$型积分直接.我们通过下面两个例子来说明这种方法.
\begin{example}\label{exam5.5.11}
  计算积分
  \[
    \int_0^\infty\frac{x^{p-1}}{(1+x)^m}\dx,
  \]
  这里,$m$是正整数,$p$不是整数,$0<p<m$.
\end{example}
\begin{solution}
  取$f(z)=\frac{z^{p-1}}{(1+z)^m}$,因为$p$不是整数,所以
  \[
    z^{p-1} = \ee^{(p-1)\Log z}
  \]
  是一个多值函数.在复平面上,取正实轴作割线得一域,$z^{p-1}$在这个域中能分出单值的全纯分支. 今取定在正实轴上沿取实值的那个全纯分支,即主支:$z^{p-1}=\ee^{(p-1)\log z}$.让$f(z)=
  \frac{\ee^{(p-1)\log z}}{(1+z)^m}$沿如下的闭曲线$\Gamma$积分:先沿正实轴的上沿从$\rho$到$R$($0<\rho<1<R<\infty$),再按反时针方向,沿以原点为中心、$R$为半径的圆周$\gamma_R$回到出发处,再沿正实轴的下沿从$R$到$\rho$,最后按顺时针方向沿以原点为中心、$\rho$为半径的圆周$\gamma_\rho$回到原来的出发处(图 \ref{fig5.4}).在正实轴上沿,有
  \begin{figure}[!ht]
    \centering
    \begin{tikzpicture}[thick,every node/.style={inner sep=2pt},
      >={Stealth[width=3pt]}]
      \draw(4:0.8)node[above right]{$\rho$}--++(2.215,0)node[above right]{$R$}--++(-2.229,0)
         (3:0.8)arc(3:356:0.8)--++(2.215,0);
      \draw(1:3)arc(1:358.7:3);
      \draw[->,very thin](30:3)arc(30:110:3);
      \draw[->,very thin](0,-0.8)--++(-0.1,0);
      \draw(140:1)node{$\gamma_\rho$}(70:3.2)node{$\gamma_R$};
      \fill(0,0)circle(1pt)node[below]{$O$};
    \end{tikzpicture}
    \caption{\label{fig5.4}}
  \end{figure}
  \[
    f(z) = \frac{\ee^{(p-1)\log x}}{(1+x)^m} = \frac{x^{p-1}}{(1+x)^m};
  \]
  在正实轴下沿,由于
  \[
    \log z = \log|z|+2\pi\ii,
  \]
  所以
  \[
    \ee^{(p-1)\log z} = \ee^{(p-1)(\log x+2\pi\ii)} = x^{p-1}\ee^{(p-1)2\pi\ii} = \ee^{2p\pi\ii}x^{p-1},
  \]
  因而
  \[
    f(z) = \ee^{2p\pi\ii}\frac{x^{p-1}}{(1+x)^m}.
  \]
  显然,$f$在由$\Gamma$围成的域中只有一个$m$阶极点$z=-1$.由残数定理,有
  \begin{equation}\label{eq5.5.7}
    \begin{gathered}
      \int_\rho^R\frac{x^{p-1}}{(1+x)^m}\dx+\int\limits_{\gamma_R}\frac{z^{p-1}}{(1+z)^m}\dz
      + \ee^{2p\pi\ii}\int_R^\rho\frac{x^{p-1}}{(1+x)^m}\dx\\
      + \int\limits_{\gamma_\rho^-}\frac{z^{p-1}}{(1+z)^m}\dz=
      2\pi\ii\Res\bigg(\frac{z^{p-1}}{(1+z)^m},-1\bigg).
    \end{gathered}
  \end{equation}
  当$z\in\gamma_R$时,$z=R\ee^{\ii\theta},\log z=\log R+\ii\theta$,所以
  \[
    \frac{|z^{p-1}|}{|1+z|^m}=\frac{|\ee^{(p-1)\log z}|}{|1+z|^m}\le\frac{R^{p-1}}{(R-1)^m}.
  \]
  同样道理,当$z\in\gamma_\rho$时,有
  \[
    \frac{|z^{p-1}|}{|1+z|^m}\le\frac{\rho^{p-1}}{(1-\rho)^m}.
  \]
  于是
  \begin{align*}
    & \bigg| \int\limits_{\gamma_R}\frac{z^{p-1}}{(1+z)^m} \dz \bigg|
    \le \frac{R^{p-1}}{(R-1)^m}2\pi R=2\pi\frac{R^{p}}{(R-1)^m}\to0\;\mbox{($R\to\infty$)},\\
    & \bigg|\int\limits_{\gamma_\rho}\frac{z^{p-1}}{(1+z)^m}\bigg|
    \le \frac{\rho^{p-1}}{(1-\rho)^m}2\pi\rho=2\pi\frac{\rho^{p}}{(1-\rho)^m}
    \to0 \;\mbox{($\rho\to0$)}.
  \end{align*}
  在 \eqref{eq5.5.7} 式中令$\rho\to0,R\to\infty$,即得
  \[
     (1-\ee^{2p\pi\ii})\int_0^\infty\frac{x^{p-1}}{(1+x)^m}\dx
    = 2\pi\ii\Res\bigg(\frac{z^{p-1}}{(1+z)^m},-1\bigg).
  \]

  容易算出,当$m=1$时
  \[
    \Res\bigg(\frac{z^{p-1}}{1+z},-1\bigg) = \ee^{(p-1)\pi\ii} = -\ee^{p\pi\ii};
  \]
  当$m>1$时
  \[
    \Res\bigg(\frac{z^{p-1}}{(1+z)^m},-1\bigg) = -\frac1{(m-1)!}(1-p)(2-p)\cdots
    (m-1-p)\ee^{p\pi\ii}.
  \]
  由此即得
  \begin{align*}
    & \int_0^\infty\frac{x^{p-1}}{1+x}\dx = \frac\pi{\sin p\pi}\;\mbox{($0<p<1$)},\\
    & \int_0^\infty\frac{x^{p-1}}{(1+x)^m}\dx = \frac\pi{\sin p\pi}\frac1{(m-1)!}(1-p)(2-p)\cdots(m-1-p). \qedhere
  \end{align*}
\end{solution}

上面的方法可用来计算一般的积分
\[
  \int_0^\infty f(x)x^{p-1}\dx\;\mbox{($0<p<1$)}.
\]

\begin{example}\label{exam5.5.12}
  计算积分
  \[
    \int_0^\infty\frac{\log x}{(1+x^2)^2}\dx.
  \]
\end{example}
\begin{solution}
  取函数$f(z)=\frac{\log^2z}{(1+z^2)^2}$,取围道如图 \ref{fig5.4} 所示.在正实轴的上沿,有
  \[
    f(z) = \frac{\log^2x}{(1+x^2)^2};
  \]
  在正实轴的下沿,由于$\log z=\log x+2\pi\ii$,所以
  \[
    \log^2z = (\log x+2\pi\ii)^2=\log^2x+4\pi\ii\log x-4\pi^2,
  \]
  因而
  \[
    f(z) = \frac{\log^2x}{(1+x^2)^2}+4\pi\ii\frac{\log x}{(1+x^2)^2}-4\pi^2\frac1{(1+x^2)^2}.
  \]
  $f$在$\Gamma$所围成的域中有两个$2$阶极点$z=\pm\ii$.对$f$用残数定理,得
  \begin{equation}\label{eq5.5.8}
    \begin{aligned}
      \int_\rho^R\frac{\log^2x}{(1+x^2)^2}\dx&+\int\limits_{\gamma_R}\frac{\log^2z}{(1+z^2)^2}\dz
      + \int_R^\rho\frac{\log^2x}{(1+x^2)^2}\dx\\
      & + 4\pi\ii\int_R^\rho\frac{\log x}{(1+x^2)^2}\dx-4\pi^2\int_R^\rho\frac{\dx}{(1+x^2)^2}
      + \int\limits_{\gamma_\rho^-}\frac{\log^2z}{(1+z^2)^2}\dz\\
      & = 2\pi\ii\bigg[\Res\bigg(\frac{\log^2z}{(1+z^2)^2},\ii\bigg)
      + \Res\bigg(\frac{\log^2z}{(1+z^2)^2},-\ii\bigg)\bigg].
    \end{aligned}
  \end{equation}
  \eqref{eq5.5.8} 式左端的第一个和第三个积分互相抵消了. $\gamma_R$和$\gamma_\rho$上两个积分的估计与例 \ref{exam5.5.11} 一样:
  \begin{align*}
    & \begin{aligned}
        \bigg|\int\limits_{\gamma_R}\frac{\log^2z}{(1+z^2)^2}\dz\bigg| &
        =  \bigg|\int_0^{2\pi}\frac{(\log R+\ii\theta)^2}{(1+R^2\ee^{2\ii\theta})^2}R\ii\ee^{\ii\theta}\dif \theta\bigg|\\
        & \le 2\pi R\frac{(\log R+2\pi)^2}{(R^2-1)^2}\to0\;\mbox{($R\to\infty$)},
      \end{aligned}\\
    & \begin{aligned}
        \bigg|\int\limits_{\gamma_\rho^-}\frac{\log^2z}{(1+z^2)^2}\dz\bigg|&
        = \bigg|\int_0^{2\pi}\frac{(\log \rho+\ii\theta)^2}{(1+\rho^2\ee^{2\ii\theta})^2}\rho\ii\ee^{\ii\theta}\dif \theta\bigg|\\
        & \le2\pi \rho\frac{(\log \rho+2\pi)^2}{(1-\rho^2)^2}\to0\;\mbox{($\rho\to0$)}.
      \end{aligned}
  \end{align*}
  直接计算残数,得
  \begin{align*}
    & \Res \bigg(\frac{\log^2z}{(1+z^2)^2},\ii\bigg)=\frac{-4\pi+\pi^2\ii}{16},\\
    & \Res \bigg(\frac{\log^2z}{(1+z^2)^2},-\ii\bigg)=\frac{12\pi-9\pi^2\ii}{16}.
  \end{align*}
  在 \eqref{eq5.5.8} 式中令$\rho\to0,R\to\infty$,并取两端的虚部,即得
  \[
    \int_0^\infty\frac{\log x}{(1+x^2)^2}\dx = -\frac\pi4.
  \]

  在计算过程中我们发现,如果取$f(z)=\frac{\log z}{(1+z^2)^2}$,则所需计算的积分将被抵消掉,这是取$f(z)=\frac{\log^2z}{(1+z^2)^2}$的原因.但若改变围道如图 \ref{fig5.5} 所示,那么取$f(z)=\frac{\log z}{(1+z^2)^2}$也是可以的.这时,$f$在$\Gamma$围成的域中只有一个$2$阶极点$z=\ii$.当$z\in[-R,-\rho]$时,$\log z=\log |x|+\ii\pi$.对$f$在$\Gamma$上应用残数定理,可得
  \begin{equation}\label{eq5.5.9}
    \begin{aligned}
      \int_{-R}^{-\rho}\frac{\log|x|}{(1+x^2)^2}\dx&+\ii\pi\int_{-R}^{-\rho}\frac{\dx}{(1+x^2)^2}
      + \int\limits_{\gamma_\rho^-}\frac{\log z}{(1+z^2)^2}\dz\\
      & + \int_\rho^R\frac{\log x}{(1+x^2)^2}\dx+\int\limits_{\gamma_R}\frac{\log z}{(1+z^2)^2}\dz\\
      & = 2\pi\ii \Res\bigg(\frac{\log z}{(1+z^2)^2},\ii\bigg).
    \end{aligned}
  \end{equation}
  \begin{figure}[!ht]
    \centering
    \begin{tikzpicture}[thick,every node/.style={inner sep=2pt},scale=1.3,
      >={Stealth[width=3pt]}]
      \draw(-3,0)node[below]{$-R$}--(-1,0)node[below]{$-\rho$}arc(180:0:1)node[below]{$\rho$}
         --(3,0)node[below]{$R$}arc(0:180:3);
      \draw[<-,very thin](0,1)node[above]{$\gamma_\rho$}--++(-0.01,0);
      \draw[->,very thin](0,3)node[above]{$\gamma_R$}(45:3)--++(-0.01,0.01);
      \fill(0,0)circle(0.7pt)node[below]{$O$};
    \end{tikzpicture}
    \caption{\label{fig5.5}}
  \end{figure}
  与上面的做法一样,可证
  \begin{align*}
    & \lim_{R\to\infty}\int\limits_{\gamma_R}\frac{\log z}{(1+z^2)^2}\dz=0,\\
    & \lim_{\rho\to0}\int\limits_{\gamma_\rho^-}\frac{\log z}{(1+z^2)^2}\dz=0,
  \end{align*}
  而
  \[
    \Res\bigg(\frac{\log z}{(1+z^2)^2},\ii\bigg) = \frac\pi8+\frac\ii4,
  \]
  在 \eqref{eq5.5.9} 式两端令$\rho\to0,R\to\infty$,得
  \[
    2\int_0^\infty\frac{\log x}{(1+x^2)^2}\dx-\ii\pi\int_0^\infty\frac{\dx}{(1+x^2)^2}
    = 2\pi\ii\bigg(\frac\pi8+\frac\ii4\bigg),
  \]
  两边取实部,即得
  \[
    \int_0^\infty\frac{\log x}{(1+x^2)^2}\dx = -\frac\pi4.
  \]
  与第一种方法所得的结果一样.
\end{solution}

\subsection{\mbox{$\int_a^b f(x){\normalfont\dif }x$}型积分}
我们讨论两种重要类型的有穷限积分.一种是
\[
  \int_0^{2\pi}R(\sin\theta,\cos\theta)\dif \theta
\]
类型的积分,其中,$R(X,Y)$是两个变量$X,Y$的有理函数.这种类型的积分可以化为$\int_{-\infty}^\infty f(x)\dx$型积分来讨论.事实上,因为被积函数是周期为$2\pi$的函数,所以
\[
  \int_0^{2\pi}R(\sin\theta,\cos\theta)\dif \theta = \int_{-\pi}^{\pi}R(\sin\theta,\cos\theta)\dif \theta.
\]
作变换$t=\tan\frac\theta2$,那么
\begin{align*}
  & \sin\theta = \frac{2t}{1+t^2},\\
  & \cos\theta = \frac{1-t^2}{1+t^2},\\
  & \dif \theta = \frac{2\dif t}{1+t^2},
\end{align*}
于是
\[
  \int_0^{2\pi}R(\sin\theta,\cos\theta)\dif \theta
  = 2\int_{-\infty}^\infty R\bigg(\frac{2t}{1+t^2},\frac{1-t^2}{1+t^2}\bigg)
  \frac1{1+t^2}\dif t.
\]
右端积分中的被积函数是$t$的有理函数,这是刚讨论过的积分.

计算这种积分的另外一种方法是把它化为单位圆周上的积分.设$z=\ee^{\ii\theta}$,那么
\begin{align*}
  & \cos\theta = \frac12(\ee^{\ii\theta}+\ee^{-\ii\theta}) = \frac12\bigg(z+\frac1z\bigg),\\
  & \sin\theta = \frac1{2\ii}(\ee^{\ii\theta}-\ee^{-\ii\theta}) = \frac1{2\ii}\bigg(z-\frac1z\bigg),\\
  & \dif \theta = \frac{1}{\ii z}\dz,
\end{align*}
于是
\[
  \int_0^{2\pi}R(\sin\theta,\cos\theta)\dif \theta=
  \int\limits_{|z|=1}R\bigg(\frac1{2\ii}\bigg(z-\frac1z\bigg),
  \frac12\bigg(z+\frac1z\bigg)\bigg)\frac1{\ii z}\dz.
\]
右端积分中的被积函数是$z$的有理函数,积分在单位圆周上进行,因而可用残数定理来计算.

\begin{example}\label{exam5.5.13}
  计算积分
  \[
    \int_0^{2\pi}\frac{\dif \theta}{3+\cos\theta+2\sin\theta}.
  \]
\end{example}
\begin{solution}
  令$z=\ee^{\ii\theta}$,则
  \[
    \int_0^{2\pi}\frac{\dif \theta}{3+\cos\theta+2\sin\theta}
    =2\int\limits_{|z|=1}\frac{\dz}{(\ii+2)z^2+6\ii z+\ii-2}.
  \]
  右端积分中的被积函数有两个$1$阶极点
  \[
    a_1 = -\frac{1+2\ii}5, \quad a_2 = -1-2\ii,
  \]
  但只有$a_1$在单位圆内,被积函数在$a_1$处的残数为$\frac1{4\ii}$,因而
  \begin{equation*}
    \int_0^{2\pi}\frac{\dif \theta}{3+\cos\theta+2\sin\theta}
    = 4\pi\ii\cdot\frac1{4\ii} = \pi. \qedhere
  \end{equation*}
\end{solution}

用类似的方法可以计算积分
\[
  \int_0^{2\pi}R(\sin n\theta,\cos n\theta)\dif \theta,
\]
这是因为
\[
  \int_0^{2\pi}R(\sin n\theta,\cos n\theta)\dif \theta=
  \int\limits_{|z|=1}R\bigg(\frac1{2\ii}\bigg(z^n-\frac1{z^n}\bigg),
  \frac12\bigg(z^n+\frac1{z^n}\bigg)\bigg)\frac1{\ii z}\dz.
\]
这里,$n$是整数.

如果要计算积分
\[
  \int_0^{2\pi}R(\sin \theta,\cos \theta)\cos n\theta\dif \theta
\]
或
\[
  \int_0^{2\pi}R(\sin \theta,\cos \theta)\sin n\theta\dif \theta,
\]
则先利用公式
\begin{equation}\label{eq5.5.10}
  \int_0^{2\pi}R(\sin\theta,\cos\theta)\ee^{\ii n\theta}\dif \theta =
  \int\limits_{|z|=1}R\bigg(\frac1{2\ii}\bigg(z-\frac1z\bigg),
  \frac12\bigg(z+\frac1z\bigg)\bigg)\frac{z^{n-1}}{\ii}\dz.
\end{equation}
算出左端的积分,然后取实部或虚部,即得上述两个积分.

另一种重要类型的有穷限积分是
\[
  \int_a^b(x-a)^r(b-x)^sf(x)\dx,
\]
这里,$-1<r,s<1$,且$r+s=-1,0$或$1$. 对于这种积分,有下面的计算公式.

\begin{theorem}\label{thm5.5.14}
  设$f$在$\MC$中除去$a_1,\cdots,a_n$外是全纯的,$a_1,\cdots,a_n$都不在区间$[a,b]$上;设$-1<r,s<1,s\ne0$,且$r+s$是整数.如果
  \[
    \lim_{z\to\infty}z^{r+s+1}f(z) = A\ne\infty,
  \]
  那么
  \begin{equation}\label{eq5.5.11}
    \int_a^b(x-a)^r(b-x)^sf(x)\dx = -\frac{A\pi}{\sin s\pi}+\frac\pi{\ee^{-s\pi\ii}\sin
    s\pi}\sum_{k=1}^n\Res(F,a_k),
  \end{equation}
  这里,$F(z)=(z-a)^r(b-z)^sf(z)$.
\end{theorem}

证明这个定理还需要一个与引理 \ref{lemma5.5.9} 类似的引理:
\begin{lemma}\label{lemma5.5.15}
  设$f$在
  \[
    G = \{z=\rho\ee^{\ii\theta}:\rho\ge R_0,\theta_0\le\theta\le\theta_0+\alpha\}
  \]
  中连续,如果$\lim_{z\to\infty}zf(z)=A$,那么
  \[
    \lim_{\rho\to\infty}\int\limits_{\gamma_\rho}f(z)\dz=\ii A\alpha,
  \]
  这里,$\gamma_\rho=\{z=\rho\ee^{\ii\theta}:\theta_0\le\theta\le\theta_0+\alpha\}$,它的方向是沿着辐角增加的方向.
\end{lemma}

证明的方法与引理 \ref{lemma5.5.9} 完全一样,留给读者作练习.

\textbf{定理 \ref{thm5.5.14} 的证明}\quad 联结$a$和$b$,我们证明在线段$[a,b]$外部,$F(z)=(z-a)^r(b-z)^sf(z)$能分出单值全纯的分支.

事实上,记$z-a=\rho_1\ee^{\ii\theta_1},z-b=\rho_2\ee^{\ii\theta_2}$,当$z$沿线段$[a,b]$外部的任意简单闭曲线转一圈时,$z-a$和$z-b$的辐角都要增加$2\pi$,$(z-a)^r(z-b)^s$的值由原来的$\rho_1^r\rho_2^s\ee^{\ii(r\theta_1+s\theta_2)}$变为
\[
  \rho_1^r\rho_2^s\ee^{\ii(r\theta_1+s\theta_2)+2\pi(r+s)\ii} =
  \rho_1^r\rho_2^s\ee^{\ii(r\theta_1+s\theta_2)},
\]
等式成立是因为$r+s$是整数,这就是说$F(z)$的值不变.

现取定在$[a,b]$上岸
\[
  \arg(z-a) = 0,\,\arg(b-z) = 0
\]
的一支来讨论.取$R$充分大,$\varepsilon$充分小,使得由圆周$\Gamma=\{z:|z|=R\}$的内部以及圆周
$\gamma_1=\{z:|z-a|=\varepsilon\}$和圆周$\gamma_2=\{z:|z-b|=\varepsilon\}$的外部所构成的域$D$包含$f$的全部奇点$a_1,\cdots,a_n$(见图 \ref{fig5.6}). 在域$D$上对函数$F$用残数定理,得
\begin{figure}[!ht]
  \centering
    \begin{tikzpicture}[thick,every node/.style={inner sep=2pt},
      >={Stealth[width=3pt]}]
      \fill(0,0)circle(1pt)(-1.5,0)circle(1pt)node[below]{$a$}(1.7,0)circle(1pt)
        node[below]{$b$};
      \draw(0,0)node[below=1pt]{$O$}circle(3);
      \draw[very thin,->](3,0)arc(0:45:3)node[above right]{$\Gamma$};
      \draw([shift={(7:0.5)}]-1.5,0)arc(7:353:0.5)
        --node[below,pos=0.2]{$l_2$}++(2.222,0);
      \draw([shift={(-173:0.5)}]1.7,0)arc(-173:173:0.5)
        --node[above,pos=0.8]{$l_1$}++(-2.222,0);
      \draw[->,very thin]([shift={(150:0.5)}]-1.5,0)--++(60:0.12);
      \draw[->,very thin]([shift={(90:0.5)}]1.7,0)--++(0.1,0);
      \draw[->,very thin]([shift={(7:0.5)}]-1.5,0)--++(1,0);
      \draw[->,very thin]([shift={(-173:0.5)}]1.7,0)--++(-1,0);
      \draw(-1.5,-0.7)node{$\gamma_1$}(1.7,-0.7)node{$\gamma_2$};
   \end{tikzpicture}
   \caption{\label{fig5.6}}
\end{figure}
\begin{equation}\label{eq5.5.12}
  \begin{gathered}
    \int\limits_\Gamma F(z)\dz+\int\limits_{\gamma_1}F(z)\dz+\int\limits_{l_1}F(z)\dz
    + \int\limits_{\gamma_2}F(z)\dz+\int\limits_{l_2}F(z)\dz\\
    = 2\pi\ii\sum_{k=1}^n\Res(F,a_k),
  \end{gathered}
\end{equation}
这里,$l_1,l_2$分别是$[a,b]$上、下岸上的一段. 当$z\in l_1$时,$\arg(z-a)=0,\arg(b-z)=0$,所以
\begin{align*}
  & (z-a)^r = \ee^{r\log(z-a)} = \ee^{r\log|z-a|}=\ee^{r\log(x-a)} = (x-a)^r,\\
  & (b-z)^s = \ee^{s\log(b-z)} = \ee^{s\log|b-z|}=\ee^{s\log(b-x)} = (b-x)^s.
\end{align*}
当$z\in l_2$时,$\arg(z-a)=0,\arg(b-z)=-2\pi$,所以,
\begin{align*}
  & (z-a)^r = (x-a)^r,\\
  & (b-z)^s = \ee^{s\big(\log(b-x)+\ii\arg(b-z)\big)} = \ee^{-2s\pi\ii}(b-x)^s.
\end{align*}
于是,\eqref{eq5.5.12} 式可写为
\begin{equation}\label{eq5.5.13}
  \begin{aligned}
    \int\limits_\Gamma F(z)\dz&+\int\limits_{\gamma_1}F(z)\dz+\int\limits_{\gamma_2}F(z)\dz\\
    & + (1-\ee^{-2s\pi\ii})\int_{a+\varepsilon}^{b-\varepsilon}(x-a)^r(b-x)^sf(x)\dx\\
    & = 2\pi\ii\sum_{k=1}^n\Res(F,a_k).
  \end{aligned}
\end{equation}
因为$-1$的辐角取$-\pi$,所以
\begin{align*}
  \lim_{z\to\infty}zF(z)&=\lim_{z\to\infty}z(z-a)^r(b-z)^sf(z)\\
  & = \ee^{-s\pi\ii}\lim_{z\to\infty}z^{r+s+1}f(z)=\ee^{-s\pi\ii}A,
\end{align*}
故由引理 \ref{lemma5.5.15} 得
\[
  \lim_{R\to\infty}\int\limits_{\Gamma}F(z)\dz = 2\pi\ii\ee^{-s\pi\ii}A.
\]
由于$r+1>0,s+1>0$,所以
\begin{align*}
  & \lim_{z\to a}(z-a)F(z) = \lim_{z\to a}(z-a)^{r+1}(b-z)^sf(z) = 0,\\
  & \lim_{z\to b}(b-z)F(z) = \lim_{z\to b}(z-a)^{r}(b-z)^{s+1}f(z) = 0,
\end{align*}
故由引理 \ref{lemma5.5.9} 得
\begin{align*}
  & \lim_{\varepsilon\to0}\int\limits_{\gamma_1}F(z)\dz = 0,\\
  & \lim_{\varepsilon\to0}\int\limits_{\gamma_2}F(z)\dz = 0.
\end{align*}
在 \eqref{eq5.5.13} 式中令$R\to\infty,\varepsilon\to0$,即得
\begin{align*}
  \int_a^b(x-a)^r(b-x)^sf(x)\dx&=-\frac{2\pi\ii\ee^{-s\pi\ii}A}{1-\ee^{-2s\pi\ii}}
  + \frac{2\pi\ii}{1-\ee^{-2s\pi\ii}}\sum_{k=1}^n\Res(F,a_k)\\
  & = -\frac{\pi A}{\sin s\pi}+\frac\pi{\ee^{-s\pi\ii}\sin
    s\pi}\sum_{k=1}^n\Res(F,a_k).
\end{align*}
这就是要证明的公式 \eqref{eq5.5.11}.\hfill$\Box$

\begin{example}\label{exam5.5.16}
  计算积分
  \[
    \int_{-1}^1\frac{\dx}{\sqrt[\leftroot{-1}\uproot{2}3]{(1+x)^2(1-x)}}.
  \]
\end{example}
\begin{solution}
  题中,$r=-\frac23,s=-\frac13,r+s=-1$,是一个整数,$f(z)\equiv1$,所以
  \[
    \lim_{z\to\infty}z^{r+s+1}f(z) = 1.
  \]
  由公式 \eqref{eq5.5.11} 即得
  \begin{equation*}
    \int_{-1}^1\frac{\dx}{\sqrt[\leftroot{-1}\uproot{2}3]{(1+x)^2(1-x)}} = \frac2{\sqrt3}\pi. \qedhere
  \end{equation*}
\end{solution}
\begin{example}\label{exam5.5.17}
  计算积分
  \[
    \int_0^1\frac{\sqrt[\leftroot{-1}\uproot{2}3]{x^2(1-x)}}{(1+x)^3}\dx.
  \]
\end{example}
\begin{solution}
  题中,$r=\frac23,s=\frac13,r+s=1,f(z)=\frac1{(1+z)^3}$,因而
  \[
    \lim_{z\to\infty}z^{r+s+1}f(z) = \lim_{z\to\infty}\frac{z^2}{(1+z)^3} = 0.
  \]
  $f$在全平面上只有一个$3$阶极点$z=-1$,于是由公式 \eqref{eq5.5.11} 即得
  \begin{equation}\label{eq5.5.14}
    \int_0^1\frac{\sqrt[\leftroot{-1}\uproot{2}3]{x^2(1-x)}}{(1+x)^3}\dx
    = \frac\pi{\sin\frac\pi3}\ee^{\frac\pi3\ii}\Res
    \bigg(\frac{z^{\frac23}(1-z)^{\frac13}}{(1+z)^3},-1\bigg).
  \end{equation}
  根据命题 \ref{prop5.4.2},有
  \begin{equation}\label{eq5.5.15}
    \Res\bigg(\frac{z^{\frac23}(1-z)^{\frac13}}{(1+z)^3},-1\bigg) = \frac12
    \lim_{z\to-1}\ddd{}z\big\{z^{\frac23}(1-z)^{\frac13}\big\}.
  \end{equation}
  易知
  \[
    \ddd{}z\big\{z^{\frac23}(1-z)^{\frac13}\big\} = -\frac29z^{-\frac43}(1-z)^{\frac13}
    -\frac49z^{-\frac13}(1-z)^{-\frac23}-\frac29(1-z)^{-\frac53}z^{\frac23},
  \]
  为了计算它在$z=-1$处的值,注意当$z=-1$时,$\arg z=\pi$,$\arg(1-z)=0$,于是
  \[
    \lim_{z\to-1}\ddd{}z\big\{z^{\frac23}(1-z)^{\frac13}\big\}
    = -\frac29\ee^{-\frac43\pi\ii}\sqrt[\leftroot{-1}\uproot{2}3]2-
    \frac49\ee^{-\frac\pi3\ii}\frac1{\sqrt[\leftroot{-1}\uproot{2}3]4}-
    \frac29\ee^{\frac{2\pi}3\ii}2^{-\frac53}.
  \]
  代入 \eqref{eq5.5.15} 式后再代入 \eqref{eq5.5.14} 式,即得
  \begin{equation*}
    \int_0^1\frac{\sqrt[\leftroot{-1}\uproot{2}3]{x^2(1-x)}}{(1+x)^3}\dx
    =\frac{\sqrt[\leftroot{-1}\uproot{2}3]2\pi}{18\sqrt3}. \qedhere
  \end{equation*}
\end{solution}

\subsection{两个特殊的积分}
上面只是大致归纳了一下用残数定理计算积分的类型,但它适用的范围还是相当有限的.这里介绍的两个积分便不能用第 \hyperlink{sec5.5.2}{2} 小节中的方法来计算.

(1){\kaishu Fresnel积分\index{J!积分!Fresnel积分}$\int_0^\infty\cos x^2\dx$和
$\int_0^\infty\sin x^2\dx$}

取函数$f(z)=\ee^{\ii z^2}$,取围道如图 \ref{fig5.7} 所示.因为$f$是整函数,由Cauchy积分定理,有
\begin{equation}\label{eq5.5.16}
  \int_0^R\ee^{\ii x^2}\dx+\int\limits_{\gamma_R}\ee^{\ii z^2}\dz+\int\limits_{\gamma_2}\ee^{\ii z^2}\dz=0.
\end{equation}
当$z\in\gamma_R$时,$z=R\ee^{\ii\theta},0\le\theta\le\frac\pi4$,所以
\[
  \big|\ee^{\ii z^2}\big| = \ee^{-R^2\sin2\theta}\le\ee^{-\frac4\pi R^2\theta},0\le\theta\le\frac\pi4.
\]
\begin{figure}[!ht]
  \centering
  \begin{tikzpicture}[thick,every node/.style={inner sep=2pt},
    >={Stealth[width=3pt]}]
    \draw(0,0)node[below]{$O$}--(4,0)node[below]{$R$}arc(0:45:4)--cycle;
    \draw[very thin,->](0,0)--(2,0)node[below]{$\gamma_1$};
    \draw[->,very thin](4,0)arc(0:25:4)node[right]{$\gamma_R$};
    \draw[->,very thin](45:4)--(45:2)node[above left]{$\gamma_2$};
    \draw(0.4,0)arc(0:45:0.4);
    \node at(24:1){$\frac\pi4$};
  \end{tikzpicture}
  \caption{\label{fig5.7}}
\end{figure}
于是,当$R\to\infty$时,有
\[
  \bigg|\int\limits_{\gamma_R}\ee^{\ii z^2}\dz\bigg|
  \le\int_0^{\frac\pi4}\ee^{-\frac4\pi R^2\theta}R\dif \theta
  = \frac\pi{4R}(1-\ee^{-R^2})\to0.
\]
当$z\in\gamma_2$时,$z=r\ee^{\ii\frac\pi4},0\le r\le R$,所以
\[
  \int\limits_{\gamma_2}\ee^{\ii z^2}\dz = -\ee^{\ii\frac\pi4}\int_0^R\ee^{-r^2}\dif r.
\]
在 \eqref{eq5.5.16} 式中令$R\to\infty$,即得
\begin{equation}\label{eq5.5.17}
  \int_0^\infty\ee^{\ii x^2}\dx=\ee^{\ii\frac\pi4}\int_0^\infty\ee^{-r^2}\dif r
  =\frac{\sqrt\pi}2\ee^{\ii\frac\pi4}.
\end{equation}
这里,我们已经利用了已知的概率积分
\[
  \int_0^\infty\ee^{-r^2}\dif r = \frac{\sqrt\pi}2.
\]
在 \eqref{eq5.5.17} 式两边分别取实部和虚部,即得
\[
  \int_0^\infty\cos x^2\dx = \int_0^\infty\sin x^2\dx = \frac12\sqrt{\frac\pi2}.
\]

大家不难利用计算这两个积分的方法算出
\[
  \int_0^\infty\cos x^n\dx\;\mbox{($n>1$)}
\]
和
\[
  \int_0^\infty\sin x^n\dx\;\mbox{($n>1$)}.
\]

(2){\kaishu  Poisson积分$\int_0^\infty\ee^{-ax^2}\cos bx\dx$($a>0$) }
\index{J!积分!Poisson积分}

取函数$f(z)=\ee^{-az^2}$,取围道如图 \ref{fig5.8} 所示. 因为$f$是整函数,由Cauchy积分定理,有
\begin{equation}\label{eq5.5.18}
  \int_{-R}^R\ee^{-a x^2}\dx+\int\limits_{\gamma_1}\ee^{-az^2}\dz
  + \int\limits_{\gamma_2}\ee^{-az^2}\dz+\int\limits_{\gamma_3}\ee^{-az^2}\dz=0.
\end{equation}
\begin{figure}[!ht]
  \centering
  \begin{tikzpicture}[thick,every node/.style={inner sep=2pt},
    >={Stealth[width=3pt]}]
    \draw[->](-3,0)--(-2,0)node[below]{$-R$}--(0,0)node[below right]{$O$}
      --(2,0)node[below]{$R$}--(3,0);
    \draw[->](0,-0.5)--(0,3);
    \draw(2,0)--(2,2)--(-2,2)--(-2,0);
    \draw[->,very thin](2,0)--(2,1)node[right]{$\frac{b}{2a}$}node[left]{$\gamma_1$};
    \draw[->,very thin](2,2)node[above]{$R+\frac b{2a}\ii$}--(0.8,2)node[below]{$\gamma_2$};
    \draw[->,very thin](-2,2)
      node[above]{$-R+\frac b{2a}\ii$}--(-2,1)node[right]{$\gamma_3$};
  \end{tikzpicture}
  \caption{\label{fig5.8}}
\end{figure}
当$z\in\gamma_1$时,$z=R+\ii y,0\le y\le\frac b{2a}$,所以
\[
  \bigg|\int\limits_{\gamma_1}\ee^{-az^2}\dz\bigg|
  \le\int_0^{\frac b{2a}}\ee^{-a(R^2-y^2)}\dy
  \le\ee^{-aR^2}\cdot\ee^{a\big(\frac b{2a}\big)^2}\cdot\frac b{2a}\to0\;\mbox
  {($R\to\infty$)}.
\]
同样道理,有
\[
  \int\limits_{\gamma_3}\ee^{-az^2}\dz\to0\;\mbox{($R\to\infty$)}.
\]
当$z\in\gamma_2$时,$z=x+\frac b{2a}\ii,-R\le x\le R$,所以
\begin{align*}
  \int\limits_{\gamma_2}\ee^{-az^2}\dz&=-\int_{-R}^R\ee^{-a\big(x^2-\frac{b^2}{4a^2}
  + \frac bax\ii\big)}\dx\\
  & = -\ee^{\frac{b^2}{4a}}\int_{-R}^R\ee^{-ax^2}\ee^{-bx\ii}\dx
  = -\ee^{\frac{b^2}{4a}}\int_{-R}^R\ee^{-ax^2}\cos bx\dx.
\end{align*}
在 \eqref{eq5.5.18} 式中令$R\to\infty$,即得
\[
  \int_{-\infty}^\infty\ee^{-ax^2}\dx-\ee^{\frac{b^2}{2a}}\int_{-\infty}^\infty
  \ee^{-ax^2}\cos bx\dx=0.
\]
由概率积分可得
\[
  \int_{-\infty}^\infty\ee^{-ax^2}\dx=\sqrt{\frac\pi a},
\]
所以
\[
  \int_0^\infty\ee^{-ax^2}\cos bx\dx=\frac12\sqrt{\frac\pi a}\ee^{-\frac{b^2}{4a}}.
\]

\begin{xiti}
  \item 利用残数定理和Cauchy积分定理计算下列积分(后面的括号中为正确答案):
  \begin{enuma}
    \item \makebox[0.65\textwidth][l]{$\int_0^\infty\frac{x^2+1}{x^4+1}\dx$;} $\bigg(
          \frac{\sqrt2}2\pi\bigg)$
    \item \makebox[0.65\textwidth][l]{$\int_{-\infty}^\infty\frac{x^2-x+2}{x^4+10x^2+9}\dx$;}
       $\bigg(\frac5{12}\pi\bigg)$
    \item \makebox[0.65\textwidth][l]{$\int_0^\infty\frac{x^2}{x^4+6x^2+13}\dx$;}
       $\left(\frac\pi4\sqrt{\frac{\sqrt{13}-3}2}\right)$
    \item \makebox[0.65\textwidth][l]{$\int_0^{2\pi}\frac1{a+b\cos\theta}\dif \theta$  ($0<b<a$);} $\bigg(\frac{2\pi}{\sqrt{a^2-b^2}}\bigg)$
    \item \makebox[0.65\textwidth][l]{$\int_0^{\frac\pi2}\frac1{a+\sin^2\theta}\dif \theta$($a>0$);} $\bigg(\frac\pi2\frac1{\sqrt{a(a+1)}}\bigg)$
    \item \makebox[0.65\textwidth][l]{$\int_{-\infty}^\infty\frac{x\cos x}{x^2-2x+10}\dx$;} $\bigg(\frac{\cos1-3\sin1}{3\ee^3}\pi\bigg)$
    \item \makebox[0.65\textwidth][l]{$\int_0^\infty\frac{x\sin ax}{x^2+b^2}\dx$($a,b>0$);} $\bigg(\frac\pi2\ee^{-ab}\bigg)$
    \item \makebox[0.65\textwidth][l]{$\int_0^\infty\frac{\cos x}{(1+x^2)^3}\dx$;}
        $\bigg(\frac7{16}\pi\ee^{-1}\bigg)$
    \item \makebox[0.65\textwidth][l]{$\int_0^\infty\bigg(\frac{\sin x}x\bigg)^2\dx$;}
        $\bigg(\frac\pi2\bigg)$\\
        (\textbf{提示}:考虑$\frac{\ee^{2\ii z}-1}{z^2}$沿图 \ref{fig5.9} 所示的曲线上的积分.)
        \begin{figure}[!ht]
          \centering
          \begin{tikzpicture}[thick,every node/.style={inner sep=2pt},
            >={Stealth[width=3pt]}]
            \draw[->](-4,0)--(-2,0)--(0,0)node[below]{$O$}
              --(2,0)node[below]{$R$}--(4,0)node[below]{$x$};
            \draw[->](0,0)--(0,3.2)node[right]{$y$};
            \draw(0.6,0)node[below]{$r$}arc(0:180:0.6);
            \draw(2,0)arc(0:180:2);\fill(0,0)circle(1pt);
            \draw[->,very thin](-2,0)--(-1,0);
            \draw[->,very thin](0,0)--(1.2,0);
            \draw[->,very thin](2,0)arc(0:120:2);
            \draw[->,very thin](120:0.6)--++(30:0.1);
        \end{tikzpicture}
        \caption{\label{fig5.9}}
      \end{figure}
    \item \makebox[0.65\textwidth][l]{$\int_0^\infty\bigg(\frac{\sin x}x\bigg)^3\dx$;}
      $\bigg(\frac{3\pi}8\bigg)$\\
      (\textbf{提示}:考虑$\frac{\ee^{3\ii z}-3\ee^{\ii z}+2}{z^3}$沿图 \ref{fig5.9} 所示的曲线上的积分.)
    \item \makebox[0.65\textwidth][l]{$\int_0^\infty\frac{x^p}{1+x^2}\dx$($-1<p<1$);}
      $\bigg(\frac\pi{2\cos\frac{p\pi}2}\bigg)$
    \item \makebox[0.65\textwidth][l]
      {$\int_0^\infty\frac{x^p}{x^2+2x\cos\lambda+1}\dx$($-1<p<1,-\pi<\lambda<\pi$);}
       $\bigg(\frac\pi{\sin p\pi}\frac{\sin p\lambda}{\sin \lambda}\bigg)$
    \item \makebox[0.65\textwidth][l]
      {$\int_0^\infty\frac1{1+x^p}\dx$($p>1$);}
      $\bigg(\frac\pi{p\sin\frac\pi p}\bigg)$
    \item \makebox[0.65\textwidth][l]
      {$\int_{-\infty}^\infty\frac{\ee^{px}}{1+\ee^x}\dx$($0<p<1$);}
      $\bigg(\frac\pi{\sin p\pi}\bigg)$
    \item \makebox[0.65\textwidth][l]
      {$\int_0^1\frac{x^{1-p}(1-x)^p}{1+x^2}\dx$($-1<p<2$);}
      $\bigg(\frac\pi{\sin p\pi}\bigg(2^{\frac p2}\cos\frac{p\pi}4-1\bigg)\bigg)$\\
      (\textbf{提示}:考虑$\frac{z^{1-p}(1-z)^p}{1+z^2}$沿图 \ref{fig5.10} 所示的曲线上的积分.)
      \begin{figure}[!ht]
        \centering
        \begin{minipage}[b]{0.48\textwidth}
          \centering
          \begin{tikzpicture}[thick,every node/.style={inner sep=2pt},scale=1.5,
            >={Stealth[width=3pt]}]
            \draw[->](-2,0)--(-1.5,0)--(0,0)node[below left,inner sep=0.5pt]{$O$}--(1,0)node[below]{$1$}
              --(1.5,0)node[below right]{$R$}--(2,0)node[below]{$x$};
            \draw[->](0,-1.7)--(0,2)node[right]{$y$};
            \draw(0,0)circle(1.5);
            \fill(0,0)circle(0.7pt)(1,0)circle(0.7pt)(-0.3,0)circle(0.7pt)(1.5,0)circle(0.7pt);
            \draw(10:0.3)--([shift={(170:0.3)}]1,0)arc(170:-170:0.3)
              --(-10:0.3)arc(-10:-351.8:0.3);
            \draw(-0.3,0)node[below left]{$-r$};
            \draw[->,very thin](1.5,0)arc(0:120:1.5);
            \draw[->,very thin](1,0.3)--++(0.05,0);
            \draw[->,very thin](10:0.3)--++(0.2,0);
            \draw[->,very thin]([shift={(-170:0.3)}]1,0)--++(-0.2,0);
            \draw[->,very thin](140:0.3)--++(50:0.05);
          \end{tikzpicture}
          \caption{\label{fig5.10}}
        \end{minipage}
        \begin{minipage}[b]{0.48\textwidth}
          \centering
          \begin{tikzpicture}[thick,every node/.style={inner sep=2pt},scale=1.2,
            >={Stealth[width=3pt]}]
            \draw[->](0,0)node[below left]{$O$}--(0.3,0)node[below]{$r$}--(3,0)node[below]{$R$}
              --(4,0)node[below]{$x$};
            \draw(0,0.3)--(0,0)--(0.3,0);
            \draw[->](0,0)--(0,4)node[right]{$y$};
            \draw(3,0)arc(0:90:3);
            \draw[->,very thin](0,0)--(2,0);
            \draw[->,very thin](3,0)arc(0:45:3);
            \draw[->,very thin](0,3)--(0,1.5);
            \draw(0.3,0)arc(0:90:0.3);
            \draw[->,very thin](45:0.3)--++(-45:0.05);
          \end{tikzpicture}
          \caption{\label{fig5.11}}
        \end{minipage}
      \end{figure}
    \item \makebox[0.65\textwidth][l]
      {$\int_0^1\frac1{(1+x)^2}\sqrt{\frac{(1-x)^3}x}\dx$;}
      $\bigg(\bigg(1-\frac{\sqrt2}2\bigg)\pi\bigg)$
    \item \makebox[0.65\textwidth][l]
      {$\int_{-1}^1\frac{\sqrt[\leftroot{-1}\uproot{2}4]{(1-x)^3(1+x)}}{1+x^2}\dx$;}
      $\bigg(\bigg(\sqrt{2+\sqrt2}-\sqrt2\bigg)\pi\bigg)$
    \item \makebox[0.65\textwidth][l]
      {$\int_0^\infty\frac{\log x}{x^2+2x+2}\dx$;}
      $\bigg(\frac\pi8\log2\bigg)$
    \item \makebox[0.65\textwidth][l]
      {$\int_0^\infty\frac{\log^2x}{x^2+a^2}\dx$($a>0$);}
      $\bigg(\frac\pi{8a}(\pi^2+4\log^2a)\bigg)$
    \item \makebox[0.65\textwidth][l]
      {$\int_0^\infty\frac{\sqrt x\log x}{x^2+1}\dx$;}
      $\bigg(\frac{\sqrt2}4\pi^2\bigg)$
    \item \makebox[0.65\textwidth][l]
      {$\int_0^\infty\frac{\log x}{x^2-1}\dx$;}
      $\bigg(\frac{\pi^2}4\bigg)$\\
      (\textbf{提示}:考虑$\frac{\log z}{z^2-1}$沿图 \ref{fig5.11} 所示的曲线上的积分.)
    \item \makebox[0.65\textwidth][l]
      {$\int_0^1\frac1{x+1}\log\bigg(\frac{1-x}x\bigg)\dx$;}
      $\bigg(\frac12\log^22\bigg)$
    \item \makebox[0.65\textwidth][l]
      {$\int_0^\infty\log\bigg(\frac{\ee^x+1}{\ee^x-1}\bigg)\dx$;}
      $\bigg(\frac{\pi^2}4\bigg)$
    \item \makebox[0.65\textwidth][l]
      {$\int_0^\infty\frac{\sin x}{\ee^x-1}\dx$;}
      $\bigg(\frac\pi2\bigg(\frac{\ee^{2\pi}+1}{\ee^{2\pi}-1}\bigg)-\frac12\bigg)$\\
      (\textbf{提示}:考虑$\frac{\ee^{\ii z}}{\ee^z-1}$沿图 \ref{fig5.12} 所示的曲线上的积分.)
      \begin{figure}[!ht]
        \centering
        \begin{minipage}{0.48\textwidth}
          \centering
          \begin{tikzpicture}[thick,every node/.style={inner sep=2pt},
            >={Stealth[width=3pt]}]
            \draw[->](0,0)node[below left]{$O$}--(5,0)node[below]{$R$}--(6.2,0)node[below]{$x$};
            \draw(5,0)--(5,2.5)--(0,2.5)node[left]{$2\pi\ii$};
            \draw[->](0,0)--(0,3.2)node[right]{$y$};
            \draw[->,very thin](0,2.5)--(0,1.2);
            \draw[->,very thin](0,0)--(2.5,0);
            \draw[->,very thin](5,0)--(5,1.3);
            \draw[->,very thin](5,2.5)--(2.5,2.5);
            \draw(0.4,0)node[below]{$r$}arc(0:90:0.4);
            \draw(0,2.1)arc(-90:0:0.4);
            \draw[->,very thin](45:0.4)--++(-45:0.1);
            \draw[->,very thin]([shift={(-45:0.4)}]0,2.5)--++(-135:0.1);
          \end{tikzpicture}
          \caption{\label{fig5.12}}
        \end{minipage}
        \begin{minipage}{0.48\textwidth}
          \centering
          \begin{tikzpicture}[thick,every node/.style={inner sep=2pt},
             >={Stealth[width=3pt]}]
             \draw[->](0,0)node[below left]{$O$}--(5,0)node[below]{$R$}--(6.2,0)node[below]{$x$};
             \draw(5,0)--(5,2.5)--(0,2.5)node[left]{$2\pi\ii$};
             \draw[->](0,0)--(0,3.2)node[right]{$y$};
             \draw[->,very thin](0,2.5)--(0,1.9);
             \draw[->,very thin](0,0)--(2.5,0);
             \draw[->,very thin](5,0)--(5,1.3);
             \draw[->,very thin](5,2.5)--(2.5,2.5);
             \draw(0,0.95)arc(-90:90:0.3);
             \draw[->,very thin](0,1.25)--(0,0.4);
             \draw[->,very thin](0.3,1.25)--++(0,-0.1);
             \fill(0,1.25)circle(1pt)node[left]{$\pi\ii$};
          \end{tikzpicture}
          \caption{\label{fig5.13}}
        \end{minipage}
    \end{figure}
    \item \makebox[0.65\textwidth][l]
      {$\int_0^\infty\frac x{\ee^x+1}\dx$;}
      $\bigg(\frac{\pi^2}{12}\bigg)$\\
      (\textbf{提示}:考虑$\frac{z^2}{\ee^z+1}$沿图 \ref{fig5.13} 所示的曲线上的积分.)
    \item \makebox[0.65\textwidth][l]
      {$\int_0^\infty x^n\ee^{-x^{\frac14}}\sin x^{\frac14}\dx$($n=0,1,2,\cdots$);}
      $(0)$\\
     (\textbf{提示}:考虑$z^{4n+3}\ee^{-z+\ii z}$沿图 \ref{fig5.14} 所示的曲线上的积分.)
     \begin{figure}[!ht]
       \centering
       \begin{minipage}{0.48\textwidth}
         \centering
         \begin{tikzpicture}[thick,every node/.style={inner sep=2pt},
           >={Stealth[width=3pt]}]
           \draw[->](0,0)node[below left]{$O$}--(3,0)node[below]{$R$}
             --(4,0)node[below]{$x$};
           \draw(0,0.3)--(0,0)--(0.3,0);
           \draw[->](0,0)--(0,4)node[right]{$y$};
           \draw(3,0)arc(0:90:3);
           \draw[->,very thin](0,0)--(2,0);
           \draw[->,very thin](3,0)arc(0:45:3);
           \draw[->,very thin](0,3)--(0,1.5);
         \end{tikzpicture}
         \caption{\label{fig5.14}}
       \end{minipage}
       \begin{minipage}{0.48\textwidth}
         \centering
          \begin{tikzpicture}[thick,every node/.style={inner sep=2pt},
            >={Stealth[width=3pt]}]
            \draw[<->](0,4)node[right]{$y$}--(0,0)node[below left]{$O$}--(4,0)node[below]{$x$};
            \draw(3,0)node[below]{$R$}arc(0:50:3)--(0,0);
            \draw[->,very thin](0,0)--(1.5,0);
            \draw[->,very thin](25:3)--++(115:0.1);
            \draw[->,very thin](50:3)--(50:1.5);
            \draw(0.3,0)arc(0:50:0.3)(0.4,0)arc(0:50:0.4);
            \node at (24:0.6){$\varphi$};
          \end{tikzpicture}
          \caption{\label{fig5.15}}
       \end{minipage}
      \end{figure}
    \item \makebox[0.65\textwidth][l]
      {$\int_{-\infty}^\infty\ee^{-(x+\ii a)^2}\dx$;($a\in \MR$)}
      $\big(\sqrt\pi\big)$
    \item \makebox[0.65\textwidth][l]
      {$\int_0^\infty\ee^{-ax^2}\cos bx^2\dx$($a>0$);}
      $\left(\frac{\sqrt{2\pi}}4\sqrt{\frac{\sqrt{a^2+b^2}+a}{a^2+b^2}}\right)$\\
      (\textbf{提示}:考虑$\ee^{z^2(-a+\ii b)}$沿图 \ref{fig5.15} 所示的曲线上的积分,其中$\varphi$待定.)
    \item \makebox[0.65\textwidth][l]
      {$\int_0^{\frac\pi2}\log\sin\theta\dif \theta$;}
      $\left(-\frac\pi2\log2\right)$\\
      (\textbf{提示}:先求出广义积分$\int_0^{\frac\pi2}\log|\ee^{\ii\theta}-1|\dif \theta$.)
    \item $\int_0^\pi\frac{x\sin x}{1-2a\cos x+a^2}\dx$($a>0$);
      \hfill $\left(\begin{cases}
      \frac\pi a\log(1+a),&0<a\le1;\\
      \frac\pi a\log\bigg(1+\frac1a\bigg),&a>1
      \end{cases}\right)$\\
      (\textbf{提示}:考虑$\frac z{a-\ee^{-\ii z}}$沿图 \ref{fig5.16} 所示的曲线上的积分.)
      \begin{figure}[!ht]
        \centering
        \begin{tikzpicture}[thick,every node/.style={inner sep=2pt},scale=0.5,
           >={Stealth[width=3pt]}]
           \draw[->](-7,0)--(-pi,0)node[below]{$-\pi$}--(0,0)node[below]{$O$};
           \draw[->](0,0)--(pi,0)node[below]{$\pi$}--(7,0)node[below]{$x$};
           \draw[->](0,0)--(0,10)node[right]{$y$};
           \draw(-pi,0)--(pi,0);
           \draw(pi,0)--(pi,7)--(-pi,7)--(-pi,0);
           \draw[->,very thin](pi,0)--(pi,3.5);
           \draw[->,very thin](-pi,7)--(-pi,3.5);
           \draw[->](0,7)node[below left]{$R\ii$}--(-0.8,7);
        \end{tikzpicture}
        \caption{\label{fig5.16}}
      \end{figure}
    \item  \makebox[0.65\textwidth][l]
      {$\int_0^{\infty}\frac{\log(1+x^2)}{1+x^2}\dx$;}
      $(\pi\log2)$
  \end{enuma}
  \item 设$f(z)$是有理函数,在$[0,\infty)$上无极点,并且$\infty$是$f(z)$的零点. 证明:
    \[
      \int_0^\infty\frac{f(x)}{(\log x)^2+\pi^2}\dx
      =\sum_{k=1}^n\Res\bigg(\frac{f(z)}{\Log z-\pi\ii},a_k\bigg),
    \]
    其中,$a_1=-1,a_2,a_3,\cdots,a_n$是$f(z)$在$\MC$中的全部彼此不同的极点,$\Log z=\log|z|+\ii\Arg z,0<\Arg z<2\pi,z\in\MC\backslash[0,\infty)$.
\end{xiti}

\section{一般域上的Mittag--Leffler定理、Weierstrass因子分解定理和插值定理\label{sec5.6}}
\begin{definition}\label{def5.6.1}
设$D$是域,$f$在$D$上除了极点外,在其他点处都全纯,则称$f$是$D$上的\textbf{亚纯函数}.\index{F!复变函数!亚纯函数}
\end{definition}

这里定义的亚纯函数是 \ref{sec5.3} 节中所定义的亚纯函数的推广,或者说 \ref{sec5.3} 节中所定义的亚纯函数是指$\MC$上的亚纯函数.

现在,我们要问:

(1) \hypertarget{5.6.1}{} 若$\{a_n\}$是域$D$中的一列互不相同并且在$D$内部无极限点的点列,$\psi_n(z)=\sum_{j=1}^{k_n}\frac{c_{n,j}}{(z-a_n)^j}$($n=1,2,\cdots$)是给定的一列有理函数,是否存在$D$上的亚纯函数$f$,使得$f$恰以$\{a_n\}$为其极点集,并且在每个$a_n$处的Laurent展开式的主要部分恰为$\psi_n(z)$?

(2) \hypertarget{5.6.2}{} 若$\{a_n\}$是单连通域$D$中的一列互不相同并且在$D$内部无极限点的点列,$\{k_n\}$是一列自然数,是否存在$D$上的全纯函数$f$,使得$f$恰以$\{a_n\}$为其零点集,并且在每个$a_n$处的零点阶数恰为$k_n$?

(3) \hypertarget{5.6.3}{} 若$\{a_n\}$是单连通域$D$中的一列互不相同并且在$D$内部
无极限点的点列,$P_n(z)=\sum_{j=0}^{k_n}b_{n,j}(z-a_n)^j$($n=1,2,\cdots$)是给定的一列多项式,是否存在$D$上的全纯函数$f$,使得$f$在每个$a_n$处的Taylor级数的前$k_n+1$项部分和恰为$P_n(z)$?

当$\{a_n\}=\{a_1,a_2,\cdots,a_N\}$为有限个点时,问题 \hyperlink{5.6.1}{(1)} 和 \hyperlink{5.6.2}{(2)} 的答案显然是肯定的.因为在 \hyperlink{5.6.1}{(1)} 的情形下,只须令$f(z)=\sum_{n=1}^N\psi_n(z)$;在 \hyperlink{5.6.2}{(2)}  的情形下,只须令$f(z)=\prod_{n=1}^N(z-a_n)^{k_n}$即可.

至于 \hyperlink{5.6.3}{(3)} 的情形,即使$\{a_n\}=\{a_1,a_2\}$,只有两个点,也不易知道答案是否肯定,更不用说具体写出一个满足条件的$f$来了.

下面我们将证明问题 \hyperlink{5.6.1}{(1)}, \hyperlink{5.6.2}{(2)} 和  \hyperlink{5.6.3}{(3)} 的答案都是肯定的.

\begin{theorem}[(\textbf{Mittag--Leffler定理})]\label{thm5.6.2}\index{D!定理!Mittag--Leffler定理}
  设$D$是$\MC$中的域,$\{a_n\}$是$D$中的一列互不相同并且在$D$内部无极限点的点列,且
  \begin{equation}\label{eq5.6.1}
    \psi_n(z) = \sum_{j=1}^{k_n}\frac{c_{n,j}}{(z-a_n)^j}\;\mbox{($n=1,2,\cdots$)}
  \end{equation}
  是给定的一列有理函数,则必存在$D$上的亚纯函数$f$,使得$f$恰以$\{a_n\}$为其极点集,并且在每个$a_n$处的Laurent展开式的主要部分恰为$\psi_n(z)$.
\end{theorem}
\begin{proof}
  对每个$a_n$,存在圆盘$B(a_n,3\varepsilon_n)\subset D$,使得$\{B(a_n,3\varepsilon)\}$互不相交.由引理 \ref{lemma3.7.2},存在 $\varphi_n\in C^\infty(\MC)$,使得$\varphi_n(z)$在$B(a_n,\varepsilon_n)$上恒等于$1$,并且$\supp\varphi_n \subset B(a_n,2\varepsilon_n)$.令
  \begin{equation}\label{eq5.6.2}
    u(z) = \sum_{n=1}^\infty\varphi_n(z)\psi_n(z),
  \end{equation}
  则$u\in C^\infty\big(D\backslash\{a_n:n\in\MN\}\big)$,并且当$z\in B(a_n,\varepsilon_n)\backslash\{a_n\}$时,有$u(z)=\varphi_n(z)\psi_n(z)=\psi_n(z)$. 再令
  \begin{equation}\label{eq5.6.3}
    h(z) = \begin{cases}
      \pp{u(z)}{\bar z}, & z\in D\backslash\{a_n:n\in\MN\};\\
      0, & z\in\{a_n:n\in\MN\},
    \end{cases}
  \end{equation}
  注意到当$z\in B(a_n,\varepsilon_n)\backslash\{a_n\}$时,$h(z)=\pp{u(z)}{\bar z}
  =\pp{\psi_n(z)}{\bar z}=0$,由 \eqref{eq5.6.3} 式便知$h\in C^\infty(D)$. 根据定理 \ref{thm3.7.3},下面的$\bar\partial$问题
  \[
    \pp v{\bar z} = h
  \]
  有解$v\in C^\infty(D)$. 令$f=u-v$,则$f$在$D\backslash\{a_n:n\in\MN\}$上全纯,这是因为
  $\pp f{\bar z}=\pp u{\bar z}-\pp v{\bar z}=h-h=0$的缘故. 由 \eqref{eq5.6.2} 式及$v\in C^\infty(D)$,即知$f$在每个$a_n$处的Laurent展开式的主要部分为$\psi_n(z)$.
\end{proof}

从定理 \ref{thm5.6.2} 可以推出下面的
\begin{theorem}[(\textbf{Weierstrass}因子分解定理)]\label{thm5.6.3}\index{D!定理!Weierstrass因子分解定理}
  设$D$是$\MC$中的单连通域,$\{a_n\}$是$D$中的一列互不相同并且在$D$内部无极限点的点列,$\{k_n\}$是一列自然数,则必存在$D$上的全纯函数$f$,使得$f$恰以$\{a_n\}$为其零点集,并且$f$在每个$a_n$处的零点阶数恰为$k_n$.
\end{theorem}
\begin{proof}
  由定理 \ref{thm5.6.2},存在$D$上的亚纯函数$g$,使得$g$恰以$\{a_n\}$
  为其极点集,并且$g$在每个$a_n$处的Laurent展开式的主要部分恰为$\frac{k_n}{z-a_n}$.于是,在$a_n$附近有
  \begin{equation}\label{eq5.6.4}
    g(z) = \frac{k_n}{z-a_n}+g_n(z),
  \end{equation}
  这里,$g_n(z)$在$a_n$附近全纯. 固定$a_0\in D\backslash\{a_n:n\in\MN\}$,令
  \begin{equation}\label{eq5.6.5}
    F(z) = \int_{a_0}^zg(\zeta)\dif \zeta,z\in D\backslash\{a_n:n\in\MN\}.
  \end{equation}
  则对于固定的$z$,由于$D\backslash\{a_n:n\in\MN\}$中连接$a_0$和$z$的曲线的不同,所得的积分值也不同.但从 \eqref{eq5.6.4} 式可看出,两个不同的积分值之差是$2\pi\ii$的整数倍.因此,$F(z)$是$D\backslash\{a_n:n\in\MN\}$上的多值全纯函数,但$F(z)$在同一点处的任意两个函数值之差是$2\pi\ii$的整数倍.于是,$f(z)=\ee^{F(z)}$便是$D\backslash\{a_n:n\in\MN\}$上的单值全纯函数.我们可验证$f$就是要求的$D$上的全纯函数.事实上,由 \eqref{eq5.6.4} 式和 \eqref{eq5.6.5} 式可知,在$a_n$附近有
  \[
    F(z) = k_n\Log(z-a_n) + h_n(z),
  \]
  这里,$h_n(z)$在$a_n$附近全纯,于是
  \[
    f(z) = (z-a_n)^{k_n}\ee^{h_n(z)}.
  \]
  所以$f$是$D$上的全纯函数,恰以$\{a_n:n\in\MN\}$为其零点集,并且在每个$a_n$处的零点阶数为$k_n$.
\end{proof}

从定理 \ref{thm5.6.3} 和问题的可解性能得出下面的
\begin{theorem}[(\textbf{插值定理})]\label{thm5.6.4}\index{D!定理!插值定理}
  设$\{a_n\}$是单连通域$D$中的一列互不相同并且在$D$内部无极限点的点列,且
  \begin{equation}\label{eq5.6.6}
    P_n(z) = \sum_{j=0}^{k_n}b_{n,j}(z-a_n)^j\;\mbox{($n=1,2,\cdots$)}
  \end{equation}
  是给定的一列多项式,则必存在$D$上的全纯函数$f$,使得$f$在每个$a_n$处的Taylor级数的前$k_n+1$项部分和恰为$P_n(z)$.
\end{theorem}
\begin{proof}
  由定理 \ref{thm5.6.3},可取$g\in H(D)$,使得$g$恰以$\{a_n\}$为其零点集,$g$在每个$a_n$处的零点阶数恰为$k_n+1$.对每个$a_n$,存在圆盘$B(a_n,3\varepsilon_n)\subset D$,使得$\{B(a_n,3\varepsilon_n)\}$互不相交.由引理 \ref{lemma3.7.2},存在$\varphi_n\in C^\infty(\MC)$,使得$\varphi_n(z)$在$B(a_n,\varepsilon_n)$上恒等于$1$,并且$\supp\varphi_n\subset B(a_n,2\varepsilon_n)$.令
  \begin{equation}\label{eq5.6.7}
    u(z) = \begin{cases}
     \varphi_n(z)\frac{P_n(z)}{g(z)},&z\in B(a_n,3\varepsilon_n)\backslash\{a_n\},n\in\MN;\\
     0, & z\in D\backslash\bigcup_{n=1}^\infty\big(B(a_n,3\varepsilon_n)\backslash
      \{a_n\}\big),
    \end{cases}
  \end{equation}
  则$u\in C^\infty\big(D\backslash\{a_n:n\in\MN\}\big)$,并且$z\in B(a_n,\varepsilon_n)\backslash\{a_n\}$时,有$u(z)=\varphi_n(z)\frac{P_n(z)}{g(z)}
   =\frac{P_n(z)}{g(z)}$.再令
    \begin{equation} \label{eq5.6.8}
        h(z) = \begin{cases}
        \pp{u(z)}{\bar z}, & z\in D\backslash\{a_n:n\in\MN\};\\
         0, & z\in\{a_n:n\in\MN\},
      \end{cases}
    \end{equation}
  注意到当$z\in B(a_n,\varepsilon_n)\backslash\{a_n\}$时,$h(z)=\pp{u(z)}{\bar z}
  =\pp{}{\bar z}\bigg(\frac{P_n(z)}{g(z)}\bigg)=0$,由 \eqref{eq5.6.8} 式便知$h\in C^\infty(D)$. 根据定理 \ref{thm3.7.3},下面的$\bar\partial$问题
  \[
    \pp v{\bar z} = h
  \]
  有解$v\in C^\infty(D)$. 令
  \[
    f(z) = g(z)u(z) - g(z)v(z),
  \]
  则$f$在$D\backslash\{a_n:n\in\MN\}$上全纯,这是因为$\pp f{\bar z}=g\bigg(\pp u{\bar z}-\pp v{\bar z}\bigg)=g(h-h)=0$的缘故. 由 \eqref{eq5.6.7} 式和$v\in C^\infty(D)$可知$f\in H(D)$,并且在每个$a_n$附近有
  \[
    f(z)  =P_n(z) - g(z)v(z).
  \]
  若注意到$a_n$是$g(z)v(z)$的至少$k_n+1$阶零点,则知$f$在$a_n$处的Taylor级数的前$k_n+1$项部分和便是$P_n(z)$.
\end{proof}

\begin{xiti}
  \item 设$f$和$g$都是域$D$上的亚纯函数,它们的零点集和极点集相同,并且相应的零点阶数和极点阶数也相同. 证明:必存在$h\in H(D)$,使得$\frac{f(z)}{g(z)}=\ee^{h(z)},\forall z\in D$.
  \item 设$\{a_n\}$是域$D$中互不相同并且在$D$内部无极限点的点列,Laurent级数
     \[
       \psi_n(z) = \sum_{j=1}^\infty\frac{c_{n,j}}{(z-a_n)^j}\;\mbox{($n=1,2,\cdots$)}
     \]
     在$D$上除了$a_n$外的其他点处都全纯.证明:存在$f\in H\big(D\backslash
      \{a_n:n\in\MN\}\big)$,使得$f$在每个$a_n$处的Laurent展开式的主要部分恰为$\psi_n(z)$.
  \item 证明:$f$是域$D$上的亚纯函数,当且仅当存在$g,h\in H(D),h\not\equiv0$,使得$f(z)=\frac{g(z)}{h(z)},\forall z\in D$.
  \item (插值定理的推广)设$\{a_n\}$是域$D$中互不相同并且在$D$内部无极限点的点列,Laurent级数
     \[
       \varphi_n(z) = \sum_{j=-\infty}^\infty b_{n,j}(z-a_n)^j\;\mbox{($n=1,2,\cdots$)}
     \]
     在$D$上除了$a_n$外的其他点处都全纯.证明:对于任意整数列$\{k_n\}$,必存在$D\backslash
        \{a_n,n\in\MN\}$上的全纯函数$f$,使得$f$在每个$a_n$处的Laurent展开式恰为
     \[
       \sum_{j=-\infty}^{k_n}b_{n,j}(z-a_n)^j+\sum_{j=k_n+1}^\infty c_{n,j}(z-a_n)^j.
     \]
  \item 设$a_1,a_2$是域$D$中的两个不同点. Laurent级数
     \[
       \varphi_n(z)=\sum_{j=-\infty}^\infty b_{n,j}(z-a_n)^j\;\mbox{($n=1,2$)}
     \]
     在$D$上除了$a_n$外的其他点处都全纯.举例说明,通常不存在$D\backslash\{a_1,a_2\}$上的全纯函数$f$,使得$f$在$a_1$和$a_2$处的Laurent展开式分别为$\varphi_1(z)$和$\varphi_2(z)$.
  \item 证明:Weierstrass因子分解定理对于一般的域成立.
  \item 证明:插值定理对于一般的域成立.
\end{xiti}

\section{特殊域上的Mittag--Leffler定理、
Weierstrass因子分解定理和Blaschke 乘积\label{sec5.7}}
为了叙述方便起见,我们给出下面的
\begin{definition}\label{def5.7.1}
  设$\{\gamma_n\}$是一列可求长的简单闭曲线,$l_n=\int\limits_{\gamma_n}|\dz|$是$\gamma_n$的长度,$d_n=d(0,\gamma_n)=\inf_{z\in\gamma_n}|z|$是$\gamma_n$与原点$O$之间的距离. 若
  \begin{eenum}
    \item $\gamma_n$总是位于$\gamma_{n+1}$的内部,原点$O$位于$\gamma_1$的内部;
    \item $d_n\to\infty$($n\to\infty$);
    \item $\bigg\{\frac{l_n}{d_n}\bigg\}$有界.
  \end{eenum}
  则称$\{\gamma_n\}$是\textbf{正则曲线列}\index{Z!正则曲线列}.
\end{definition}

\begin{theorem}\label{thm5.7.2}
  设$\{\gamma_n\}$是正则曲线列,$f$是$\MC$上的亚纯函数,若
  \begin{eenum}
    \item $f$的全部互不相同的极点$\{a_n\}$都是$f$的$1$阶极点,记$c_n=\Res(f,a_n)$;
    \item 原点$O$不是$f$的极点;
    \item $f$在$\bigcup_{n=1}^\infty\gamma_n$上有界,
  \end{eenum}
  则
  \begin{equation}\label{eq5.7.1}
    f(z) = f(0) + \sum_{n=1}^\infty c_n\bigg(\frac1{z-a_n}+\frac1{a_n}\bigg),
  \end{equation}
  其中,\eqref{eq5.7.1} 式右端在$\MC\backslash\{a_n:n\in\MN\}$上内闭一致收敛.
\end{theorem}
\begin{proof}
  对每个$\gamma_n$,令$D_n$是由$\gamma_n$所围成的单连通域. 设$z\in\big(B(0,R)\backslash\{0\}\big)\backslash\{a_n:n\in\MN\}$,取$m$,使得$B(0,R)\subset D_m$. 考虑积分
  \[
    I_m = \frac1{2\pi\ii}\int\limits_{\gamma_m}\frac{f(\zeta)}{\zeta(\zeta-z)}\dif \zeta.
  \]
  显然,$g(\zeta)=\frac{f(\zeta)}{\zeta(\zeta-z)}$在$D_m$中的全部极点为
  \[
    0, z, a_n \in D_m,
  \]
  $g$在这些极点处的残数分别为
  \[
    -\frac{f(0)}z, \frac{f(z)}z, \frac{c_n}{a_n(a_n-z)}.
  \]
  由残数定理便知,当$z\in\big(B(0,R)\backslash\{0\}\big)\backslash\{a_n:n\in\MN\}$时,有
  \[
    I_m = -\frac{f(0)}z + \frac{f(z)}z + \sum_{a_n\in D_m}\frac{c_n}{a_n(a_n-z)},
  \]
  或
  \[
    f(z) = f(0) + \sum_{a_n\in D_m}c_n\bigg(\frac1{z-a_n}+\frac1{a_n}\bigg) + zI_m,
    z\in B(0,R)\backslash\{a_n:n\in\MN\}.
  \]
  注意到
  \[
    |zI_m|\le R\frac1{2\pi}\int\limits_{\gamma_m}\frac{|f(\zeta)|}{|\zeta|\,
    \big||\zeta|-|z|\big|}|\dif \zeta|\le\frac{R}{2\pi}\frac{Ml_m}{d_m(d_m-R)}
    \to0\;\mbox{($m\to\infty$)},
  \]
  其中,$M=\sup_{\zeta\in\bigcup\limits_{n=1}^\infty\gamma_n}|f(\zeta)|<\infty$,便知
  \[
    f(0) + \sum_{n=1}^\infty c_n\bigg(\frac1{z-a_n}+\frac1{a_n}\bigg)
  \]
  在$\MC\backslash\{a_n:n\in\MN\}$上内闭一致收敛于$f(z)$.
\end{proof}

\begin{theorem}\label{thm5.7.3}
  设$\{\gamma_n\}$是正则曲线列,$f$是整函数. 若
  \begin{eenum}
    \item $f$的全部互不相同的零点集为$\{a_n\}$,在每个$a_n$处的零点阶数为$k_n$;
    \item $f(0)\ne0$;
    \item $\frac{f'(z)}{f(z)}$在$\bigcup_{n=1}^\infty\gamma_n$上有界,
  \end{eenum}
  则
  \begin{equation}\label{eq5.7.2}
    f(z) = f(0) \ee^{\frac{f'(0)}{f(0)}z}\prod_{n=1}^\infty\bigg(1-\frac z{a_n}\bigg)
    ^{k_n}\ee^{\frac{k_n}{a_n}z},
  \end{equation}
  其中,\eqref{eq5.7.2} 式右端在$\MC\backslash\{a_n:n\in\MN\}$上内闭一致收敛.
\end{theorem}
\begin{proof}
  对$\MC$上的亚纯函数$\frac{f'(z)}{f(z)}$应用定理 \ref{thm5.7.2},并注意到
  \[
    \Res\bigg(\frac{f'(z)}{f(z)},a_n\bigg) = k_n,
  \]
  便知
  \[
    \frac{f'(0)}{f(0)} + \sum_{n=1}^\infty\bigg(\frac{k_n}{z-a_n} + \frac{k_n}{a_n}\bigg)
  \]
  在$\MC\backslash\{a_n:n\in\MN\}$上内闭一致收敛于$\frac{f'(z)}{f(z)}$. 于是
  \begin{equation}\label{eq5.7.3}
    \Log\frac{f(z)}{f(0)}=\int_0^z\frac{f'(\zeta)}{f(\zeta)}\dif \zeta = \frac{f'(0)}{f(0)}z+\sum_{n=1}^\infty\int_0^z\bigg(\frac{k_n}{\zeta-a_n}+
    \frac{k_n}{a_n}\bigg)\dif \zeta,
  \end{equation}
  其中,积分是沿$\MC\backslash\{a_n:n\in\MN\}$中连接$O$和$z$的任意可求长曲线进行的,并且 \eqref{eq5.7.3} 式右端在$\MC\backslash\{a_n:n\in\MN\}$上内闭一致收敛.因此
  \begin{equation*}
    f(z) = f(0)\ee^{\frac{f'(0)}{f(0)}z}\prod_{n=1}^\infty \ee^{\int_0^z\left(\frac{k_n}{\zeta-a_n}+
    \frac{k_n}{a_n}\right) \dif \zeta}
    = f(0)\ee^{\frac{f'(0)}{f(0)}z}\prod_{n=1}^\infty\bigg(1-\frac z{a_n}\bigg)  ^{k_n}\ee^{\frac{k_n}{a_n}z}. \qedhere
  \end{equation*}
\end{proof}

\begin{theorem}[(\textbf{Blaschke定理})]\label{thm5.7.4}\index{D!定理!Blaschke定理}
  设$\{a_n\}$是$B(0,R)\backslash\{0\}$中互不相同的点列,$\{k_n\}$是一列自然数. 若
  \[
    \sum_{n=1}^\infty k_n(R-|a_n|) < \infty,
  \]
  则
  \begin{equation}\label{eq5.7.4}
    \bigg\{\prod_{j=1}^n\bigg(\frac{R(a_j-z)}{R^2-\bar a_jz}\bigg)^{k_j}
    \bigg(\frac{|a_j|}{a_j}\bigg)^{k_j}\bigg\}
  \end{equation}
  必在$B(0,R)$上内闭一致收敛于一个全纯映射$f:B(0,R)\to B(0,1)$,使得$f$恰以$\{a_n\}$为其零点集,且$f$在每个$a_n$处的零点阶数恰为$k_n$.
\end{theorem}
\begin{proof}
  由所给条件知$\{a_n\}$在$B(0,R)$中无极限点,因而$B(0,R)\backslash\{a_n:n\in\MN\}$是域,并且
  \[
    \sum_{n=1}^\infty\frac{k_n(R^2-|a_n|^2)}{(R^2-\bar a_nz)(z-a_n)}
  \]
  在$B(0,R)\backslash\{a_n:n\in\MN\}$上内闭一致收敛于$g\in H\big(B(0,R)\backslash
  \{a_n:n\in\MN\}\big)$. 显然,$g$是$B(0,R)$上的亚纯函数,恰以$\{a_n\}$为其极点集,且在每个$a_n$处的残数为$k_n$. 令
  \begin{equation}\label{eq5.7.5}
    \begin{aligned}
      F(z) & = \int_0^zg(\zeta)\dif \zeta=\sum_{n=1}^\infty\int_0^z
             \frac{k_n(R^2-|a_n|^2)}{(R^2-\bar a_n\zeta)(\zeta-a_n)}\dif \zeta\\
           & = \sum_{n=1}^\infty k_n\int_0^z\bigg(\frac1{\zeta-a_n}+\frac{\bar a_n}{R^2
             - \bar a_n\zeta}\bigg)\dif \zeta,
    \end{aligned}
  \end{equation}
  其中,积分是沿$B(0,R)\backslash\{a_n:n\in\MN\}$中连接$O$和$z$的可求长曲线进行的,并且 \eqref{eq5.7.5} 式右端在$B(0,R)\backslash\{a_n:n\in\MN\}$上内闭一致收敛.显然,$F(z)$是$B(0,R)\backslash\{a_n:n\in\MN\}$上的多值全纯函数,但在同一点处的任意两个函数值之差为$2\pi\ii$的整数倍,故
  \[
    h(z) = \ee^{F(z)} = \prod_{n=1}^\infty\ee^{k_n\int_0^z\left(\frac1{\zeta-a_n}
    + \frac{\bar {a_n}}{R^2-\bar {a_n}\zeta}\right)\dif \zeta}
    = \prod_{n=1}^\infty\bigg(\frac{R(a_n-z)}{R^2-\bar {a_n}z}\bigg)^{k_n}
    \bigg(\frac{R}{a_n}\bigg)^{k_n}
  \]
  是$B(0,R)$上的全纯函数,$h$恰以$\{a_n\}$为其零点集,且在每个$a_n$处的零点阶数恰为$k_n$. 再注意到$\prod_{n=1}^\infty\bigg(\frac{|a_n|}{R}\bigg)^{k_n}$是一个正数,便知 \eqref{eq5.7.4} 式在$B(0,R)$上内闭一致收敛于$h(z)\prod_{n=1}^\infty\bigg(\frac{|a_n|}{R}\bigg)^{k_n}$. 最后,由于对任意$z\in B(0,R),\bigg|\frac{R(a_n-z)}{R^2-\bar {a_n}z}\bigg|<1$,若令
  \[
    f(z) = h(z)\prod_{n=1}^\infty\bigg(\frac{|a_n|}{R}\bigg)^{k_n}
    = \prod_{n=1}^\infty\bigg(\frac{R(a_n-z)}{R^2-\bar {a_n}z}\bigg)^{k_n}
    \bigg(\frac{|a_n|}{a_n}\bigg)^{k_n},
  \]
  则
  \begin{equation*}
    f\big(B(0,R)\big)\subset B(0,1). \qedhere
  \end{equation*}
\end{proof}

\begin{example}\label{exam5.7.5}
  用极点处Laurent级数的主要部分表示$\cot z-\frac1z$.
\end{example}
\begin{solution}
  $f(z)=\cot z-\frac1z$的全部极点$\pm n\pi$($n\in\MN$)都是$1$阶的,且$\Res(f,\pm n\pi)=1,f(0)=0$.

  令$\gamma_n$为以$O$为中心,以$(2n-1)\pi$为边长,并且平行于坐标轴的正方形折线,则$\{\gamma_n\}$是正则曲线列.注意到
  \begin{align*}
    & \begin{aligned}
         \bigg|\cot\bigg(\pm\bigg(n-\frac12\bigg)\pi+\ii y\bigg)\bigg|
         & = \bigg|\frac{\ee^{\pm(2n-1)\pi\ii}\ee^{-2y}+1}
             {\ee^{\pm(2n-1)\pi\ii}\ee^{-2y}-1}\bigg|\\
         & = \frac{\ee^{2|y|}-1}{\ee^{2|y|}+1}\le1,
       \end{aligned}\\
    &  \begin{aligned}
         \bigg|\cot\bigg(x\pm\ii\bigg(n-\frac12\bigg)\pi\bigg)\bigg|
         & = \bigg|\frac{1+\ee^{-\ii2x}\ee^{\pm(2n-1)\pi}}{1-\ee^{-\ii2x}\ee^{\pm(2n-1)\pi}}\bigg|\\
         & \le \frac{\ee^{(2n-1)\pi}+1}{\ee^{(2n-1)\pi}-1}\le\frac{\ee^\pi+1}{\ee^\pi-1},
       \end{aligned}
  \end{align*}
  因而$\cot z-\frac1z$在$\bigcup_{n=1}^\infty\gamma_n$上有界. 故由定理 \ref{thm5.7.2} 得
  \[
    \cot z - \frac1z = \sum_{n=1}^\infty\bigg[\bigg(\frac1{z-n\pi}+\frac1{n\pi}\bigg) +
    \bigg(\frac1{z+n\pi}-\frac1{n\pi}\bigg)\bigg]
    = \sum_{n=1}^\infty\frac{2z}{z^2-n^2\pi^2},
  \]
  或者
  \begin{equation*}
    \cot z = \frac1z+\sum_{n=1}^\infty\frac{2z}{z^2-n^2\pi^2}. \qedhere
  \end{equation*}
\end{solution}

\begin{example}
  求$\sin z$的因子分解.
\end{example}
\begin{solution}
  整函数$f(z)=\frac{\sin z}z$的全部零点$\pm n\pi$($n\in\MN$)都是$1$阶的,且$f(0)=1$. 设$\{\gamma_n\}$是例 \ref{exam5.7.5} 中给出的正则曲线列,则
  \[
    \frac{f'(z)}{f(z)} = \cot z-\frac1z
  \]
  在$\bigcup_{n=1}^\infty\gamma_n$上有界. 故由定理 \ref{thm5.7.3} 得
  \[
    \frac{\sin z}z = \prod_{n=1}^\infty\bigg[\bigg(1-\frac z{n\pi}\bigg)\ee^{
    \frac z{n\pi}}\bigg]\bigg[\bigg(1+\frac z{n\pi}\bigg)\ee^{-\frac z{n\pi}}\bigg]
    =\prod_{n=1}^\infty\bigg(1-\frac{z^2}{n^2\pi^2}\bigg),
  \]
  或者
  \begin{equation*}
    \sin z = z\prod_{n=1}^\infty\bigg(1-\frac{z^2}{n^2\pi^2}\bigg). \qedhere
  \end{equation*}
\end{solution}

\begin{xiti}
  \item 用极点处Laurent级数的主要部分表示下列$\MC$上的亚纯函数:
    \begin{tasks}(2)
      \task $\frac1{\ee^z-1}$;
      \task $\frac1{\cos z}$;
      \task $\tan z$;
      \task $\frac1{\sin z}-\frac1z$.
    \end{tasks}
  \item 将下列整函数进行因子分解:
    \begin{tasks}(2)
      \task $\ee^z-1$;
      \task $\ee^{az}-\ee^{bz}$;
      \task $\cos z$;
      \task $\cos z-\sin z$.
    \end{tasks}
  \item 设$\{a_n\}$是$B(0,R)\backslash\{0\}$中互不相同的点列,$\{b_n\}$是点列$\{a_n\}$
    关于圆周$\partial B(0,R)$的对称点列,$\{k_n\}$是自然数列.证明:若$\sum_{n=1}^\infty k_n(R-|a_n|)<\infty$,则
    \[
      \bigg\{\prod_{j=1}^n \bigg(\frac{z-a_j}{z-b_j} \bigg)^{k_j}\bigg\}
    \]
    在$B(0,R)$上内闭一致收敛于一个全纯映射$f:B(0,R)\to B(0,1)$,使得$f$恰以$\{a_n\}$为其零点集,且$f$在每个$a_n$处的零点阶数恰为$k_n$.
  \item 设$\{a_n\}$是$\MC\backslash\{0\}$中互不相同的点列,$\{k_n\}$是自然数列. 证明:若$\sum_{n=1}^\infty\frac{k_n}{|a_n|}$收敛,则
      \[
        \bigg\{\prod_{j=1}^n\bigg(1-\frac z{a_j}\bigg)^{k_j}\bigg\}
      \]
      在$\MC$上内闭一致收敛于一个整函数$f$,使得$f$恰以$\{a_n\}$为其零点集,且$f$在每个$a_n$处的零点阶数恰为$k_n$.\\
     (\textbf{提示}:$\sum_{n=1}^\infty\frac{k_n}{z-a_n}$在$\MC\backslash\{a_n:n\in\MN\}$上内闭一致收敛.)
  \item 设$\{a_n\}$是$\MC\backslash\{0\}$中互不相同的点列,$\{k_n\}$是自然数列.证明:若存在自然数$m$,使得$\sum_{n=1}^\infty\frac{k_n}{|a_n|^{m+1}}$收敛,则
      \[
        \bigg\{\prod_{j=1}^n\bigg[\bigg(1-\frac z{a_j}\bigg)\ee^{\frac z{a_j} + \frac12\big(\frac z{a_j}\big)^2 + \cdots + \frac1m\big(\frac z{a_j}\big)^m}\bigg]^{k_j}\bigg\}
      \]
      在$\MC$上内闭一致收敛于一个整函数$f$,使得$f$恰以$\{a_n\}$为其零点集,且$f$在每个$a_n$处的零点阶数恰为$k_n$.\\
      (\textbf{提示}:$\sum_{n=1}^\infty k_n\bigg(\frac1{z-a_n}+\frac1{a_n}
      +\frac z{a_n^2}+\cdots+\frac{z^{m-1}}{a_n^m}\bigg)$在$\MC\backslash\{a_n:n\in\MN\}$
      上内闭一致收敛.)
  \item 设$\{a_n\}$是$\MC\backslash\{0\}$中互不相同的点列,$\{k_n\}$是自然数列.证明:若存在自然数列$\{p_n\}$,使得全纯函数项级数
      \[
        \sum_{m=1}^\infty k_n\bigg(\frac z{a_n}\bigg)^{p_n+1}
      \]
      在$\MC$上绝对收敛,则
      \[
        \bigg\{\prod_{j=1}^n\bigg[\bigg(1-\frac z{a_j}\bigg)\ee^{\frac z{a_j}+\frac12\big(\frac z{a_j}\big)^2+\cdots+\frac1{p_j}\big(\frac z{a_j}\big)^{p_j}}\bigg]^{k_j}\bigg\}
      \]
      在$\MC$上内闭一致收敛于一个整函数$f$,使得$f$恰以$\{a_n\}$为其零点集,且$f$在每个$a_n$处的零点阶数恰为$k_n$.\\
      (\textbf{提示}:$\sum_{n=1}^\infty k_n\bigg(\frac1{z-a_n}+\frac1{a_n}
      +\frac z{a_n^2}+\cdots+\frac{z^{p_n-1}}{a_n^{p_n}}\bigg)$在$\MC\backslash\{a_n:n\in\MN\}$
      上内闭一致收敛.)
  \item 设$\{a_n\}$是$\MC\backslash\{0\}$中互不相同的点列,$\{k_n\}$是自然数列.证明:若$\lim_{n\to\infty}|a_n|=\infty$,则
      \[
        \bigg\{\prod_{j=1}^n\bigg[\bigg(1-\frac z{a_j}\bigg)\ee^{\frac z{a_j}+\frac12\big(\frac z{a_j}\big)^2+\cdots+\frac1{jk_j}\big(\frac z{a_j}\big)^{jk_j}}\bigg]^{k_j}\bigg\}
      \]
      在$\MC$上内闭一致收敛于一个整函数$f$,使得$f$恰以$\{a_n\}$为其零点集,且$f$在每个$a_n$处的零点阶数恰为$k_n$.
  \item 设$f$是非常数的整函数,$f$在$\MC\backslash\{0\}$中的互不相同的零点集是$\{a_n\}$,其相应的零点阶数为$\{k_n\},k_0=\Res\bigg(\frac{f'(z)}{f(z)},0\bigg)$. 证明:一定存在整函数$g$,使得
      \[
        f(z) = \ee^{g(z)}z^{k_0}\prod_{n=1}^\infty\bigg[\bigg(1-\frac z{a_n}\bigg)\ee^{\frac z{a_n} + \frac1z\big(\frac z{a_n}\big)^2
        + \cdots+\frac1{nk_n}\big(\frac z{a_n}\big)^{nk_n}}\bigg]^{k_n}.
      \]
  \item 设$f$是有理函数,$\infty$是$f$的至少$2$阶的零点,并且$f$的全部互不相同的极点$a_1,a_2,\cdots,a_m$都不是整数. 证明:
    \begin{enuma}
      \item $\sum_{n=-\infty}^\infty f(n)=-\pi\sum_{k=1}^m\Res\big(f(z)\cot\pi z,a_k\big)$;
      \item $\sum_{n=-\infty}^\infty (-1)^nf(n)=-\pi\sum_{k=1}^m\Res\bigg(\frac{f(z)}{\sin\pi z},a_k\bigg)$.
    \end{enuma}
  \item 求下列级数的和:
    \begin{enuma}
      \item $\sum_{n=-\infty}^\infty\frac1{(a+n)^2},\sum_{n=-\infty}^\infty
         \frac{(-1)^n}{(a+n)^2}$($a$不是整数);
      \item $\sum_{n=0}^\infty\frac1{n^2+a^2},\sum_{n=0}^\infty\frac{(-1)^n}{n^2+a^2}$($a>0$);
      \item $\sum_{n=0}^\infty\frac{(-1)^n}{(2n+1)^3}$;
      \item $\sum_{n=1}^\infty\frac1{n^{2k}}$($k\in\MN$).
    \end{enuma}
  \item 设$f$是$B(0,R)$上非常数的有界全纯函数,$\{a_n\}$是$f$在$B(0,R)\backslash\{0\}$中的互不相同的零点集,$\{k_n\}$是相应的零点阶数. 证明:
     \[
       \sum_{n=1}^\infty k_n(R-|a_n|) < \infty.
     \]
  \item 设$f$是$B(0,R)$上非常数的有界全纯函数,$f$在$B(0,R)\backslash\{0\}$中的全部互不相同的零点集为$\{a_n\}$,相应的零点阶数为$\{k_n\},k_0=\Res\bigg(\frac{f'(z)}{f(z)},0\bigg)$.证明:一定存在$B(0,R)$上无零点的有界全纯函数$g$,使得
     \[
       f(z) = g(z)z^{k_0}\prod_{n=1}^\infty\bigg(\frac{R(a_n-z)}{R^2-\bar {a_n}z}\bigg)^{k_n}
       \bigg(\frac{|a_n|}{a_n}\bigg)^{k_n},\forall z\in B(0,R).
     \]
\end{xiti}
